%versi 2 (8-10-2016) 
\chapter{Pendahuluan}
\label{chap:intro}
   
\section{Latar Belakang}
\label{sec:label}

Pengelompokan (\textit{clustering}) merupakan prosedur untuk mencari struktur alami dari suatu kumpulan data. Proses ini melibatkan pemilihan data atau objek ke dalam kelompok (\textit{cluster}) sehingga objek-objek dalam cluster yang sama akan lebih mirip satu sama lain dibandingkan dengan objek yang berada di \textit{cluster} lain. \textit{Clustering} berguna untuk mereduksi data (mereduksi data dengan volume besar ke dalam kelompok-kelompok dengan karakteristik tertentu), mengembangkan skema klasifikasi (juga dikenal sebagai taksonomi), dan memberikan masukkan atau dukungan terhadap hipotesis mengenai struktur suatu data.

\textit{Clustering} merupakan salah satu teknik pembelajaran tak terarah (\textit{unsupervised learning}). Pembagian kelompok dalam \textit{clustering} tidak berdasarkan sesuatu yang telah diketahui sebelumnya, melainkan berdasarkan kesamaan tertentu menurut suatu ukuran tertentu \cite{raposo2014automatic}.

\textit{Document clustering} (pengelompokan dokumen) merupakan proses pengelompokan yang dilakukan terhadap suatu koleksi dokumen. Pengelompokan dokumen diterapkan dalam beberapa bidang seperti penambangan web, mesin pencari (\textit{search engine}), dan temu kembali informasi (\textit{information retrieval}) \cite{shah2012document}. Hal yang dilakukan dalam pengelompokan dokumen adalah mengukur kemiripan (\textit{similarity}) antar dokumen dan mengelompokan dokumen yang serupa. Suatu dokumen dapat terdiri dari beberapa jenis informasi seperti teks, jenis tulisan, ukuran tulisan, warna tulisan, dan gambar. 

Salah satu algoritma pengelompokan yang paling sering digunakan adalah \textit{K-means} yang dilakukan dengan cara membagi data ke dalam \textit{K} kelompok. Kelompok tersebut dibentuk dengan cara meminimalkan jarak antara titik pusat \textit{cluster} (\textit{centroid}) dengan setiap anggota \textit{cluster} tersebut. Titik pusat \textit{cluster} dicari dengan menggunakan rata-rata (\textit{mean}) dari nilai setiap anggota \textit{cluster}. Dalam hal ini, setiap anggota \textit{cluster} dimodelkan sebagai vektor dalam $n$ dimensi ($n$ merupakan banyaknya atribut). \textit{K-means} sudah terbukti efektif dalam melakukan pengelompokan  dalam situasi apapun. Namun, cara tersebut tetap saja memiliki kekurangan yaitu dapat terjebak dalam \textit{local optima} tergantung dengan pemilihan \textit{centroid} awal \cite{maulik2000genetic}.

Masalah \textit{local optima} dapat ditangani menggunakan \textit{Genetic Algorithm} (GA) yang telah terbukti efektif dalam menyelesaikan masalah pencarian dan optimasi. GA merupakan teknik pencarian heuristik tingkat tinggi yang menirukan proses evolusi yang secara alami terjadi \cite{holland1992genetic} berdasarkan prinsip \textit{survival of the fittest}. Algoritma ini dinamakan demikian karena menggunakan konsep-konsep dalam genetika sebagai model pemecahan masalahnya \cite{sivanandam2007introduction}.

Dalam GA, parameter dari \textit{search space} dikodekan dalam bentuk deretan objek yang disebut kromosom. Kumpulan kromosom tersebut lalu dikenal sebagai populasi. Pada awalnya, populasi dibangkitkan secara acak. Kemudian, akan dipilih beberapa kromosom menggunakan teknik \textit{roulette wheel selection} berdasarkan fungsi \textit{fitness}. Operasi dasar yang terinspirasi dari Ilmu Biologi seperti persilangan (\textit{crossover}) dan mutasi (\textit{mutation}) digunakan untuk membangkitkan generasi berikutnya. Proses seleksi, persilangan, dan mutasi ini berlangsung dalam jumlah generasi tertentu atau sampai kondisi akhir tercapai.

Fungsi \textit{fitness} tidak hanya berfungsi untuk menentukan seberapa baik solusi yang dihasilkan namun juga menentukan seberapa dekat solusi tersebut dengan hasil yang optimal \cite{sivanandam2007introduction}. Oleh karena itu, diperlukan fungsi \textit{fitness} yang cocok sehingga GA dapat menghasilkan keluaran yang optimal. Pada masalah \textit{clustering} menggunakan GA, maka fungsi \textit{fitness} yang digunakan harus bisa menggambarkan bahwa seluruh elemen sudah berada dalam \textit{cluster} yang terbaik dan sudah sesuai.

\section{Rumusan Masalah}
\label{sec:rumusan}
Berdasarkan latar belakang yang telah dipaparkan, rumusan masalah dari penelitian ini adalah sebagai berikut:

\begin{enumerate}
 \item Bagaimana algoritma genetik dapat digunakan untuk mengelompokkan dokumen?
 \item Bagaimana membangun perangkat lunak yang menggunakan algoritma genetik untuk dapat mengelompokkan
dokumen?
\end{enumerate}

\section{Tujuan Penelitian}
\label{sec:tujuan}
Berdasarkan rumusan masalah yang telah disebutkan, tujuan dari penelitian ini adalah sebagai berikut:

\begin{enumerate}
	\item Mempelajari algoritma genetik dan hubungannya dengan pengelompokan dokumen.
	\item Membangun perangkat lunak yang mengimplementasikan algoritma genetik untuk dapat mengelompokkan
dokumen.
\end{enumerate}

\section{Batasan Masalah}
\label{sec:batasan}
Rumusan masalah yang telah disebutkan memiliki ruang lingkup yang cukup luas. Dengan menyadari terbatasnya waktu serta kemampuan, penelitian ini difokuskan dengan memperlihatkan batasan masalah sebagai berikut:

\begin{enumerate}
	\item Jenis dokumen yang dapat diproses dengan perangkat lunak yang dibuat hanyalah \textit{Text Document} dengan ekstensi \textit{TXT}.
	\item Informasi dari dokumen yang diproses dalam pengelompokan hanya berasal dari teks yang menjadi isi dari dokumen tersebut. Gambar dan \textit{metadata} (pemilik, tanggal modifikasi) tidak diperhitungkan.
\end{enumerate}


\section{Metodologi}
\label{sec:metlit}
Langkah-langkah yang dilakukan dalam penelitian ini adalah:

\begin{enumerate}
	\item Melakukan studi literatur mengenai model ruang vektor, {\it Document Clustering} (pengelompokan dokumen), {\it Genetic Algorithm} (algoritma genetik), dan penggunaan algoritma genetik dalam pengelompokan dokumen.
	\item Mencari dokumen yang dijadikan {\it datasets}.
	\item Membuat rancangan perangkat lunak yang menggunakan algoritma genetik sebagai algoritma pengelompokan dokumen.
	\item Mengimplementasikan hasil rancangan menjadi perangkat lunak dalam bahasa pemrograman Java.
	\item Melatih dan menguji perangkat lunak dengan dokumen yang telah tersedia.
	\item Mengevaluasi hasil pengujian lalu lakukan implementasi dan pengujian kembali sampai didapatkan hasil yang sudah sesuai dengan harapan.
\end{enumerate}

\section{Sistematika Pembahasan}
\label{sec:sispem}
Dokumentasi dari penelitian ini disajikan dalam enam bab dengan sistematika pembahasan sebagai berikut:
\begin{enumerate}
	\item Bab 1 Pendahuluan\\
		Bab 1 berisi latar belakang pemilihan "Pengelompokan Dokumen berbasis Algoritma Genetika" sebagai judul dari penelitian ini. Selain itu, dibahas juga rumusan masalah, tujuan penelitian, batasan masalah, serta metodologi penelitian yang menjadi acuan dari penelitian ini.
	\item Bab 2 Landasan Teori\\
		Bab 2 memuat landasan teori yang digunakan dalam penelitian ini. Konsep-konsep yang dibahas yaitu pengelompokan, \textit{local optimum}, K-means, algoritma genetika beserta seluruh operasinya, model ruang vektor, pembobotan term yang terdiri dari bobot frekuensi dan bobot TF-IDF, metrik \textit{Intracluster} untuk mengukur minerja metode \textit{clustering}, dan rvaluasi hasil pengelompokan menggunakan \textit{purity}.
	\item Bab 3 Analisis\\		
		Bab 3 memuat hasil analisis berdasarkan landasan teori. Hasil analisis yang ditulis pada bab 3 antara lain analisis \textit{dataset}, representasi dokumen, modifikasi terhadap model ruang vektor beserta metode pembobotannya, representasi kromosom, fungsi \textit{fitness}, dan operasi genetik lainnya.
	\item Bab 4 Perancangan\\
		Bab 4 memuat hasil perancangan berdasarkan hasil analisis pada bab 3. Terdapat tiga bagian dalam bab perancangan yaitu Kebutuhan masukan dan keluaran, perancangan kelas, dan perancangan antarmuka pengguna.
	\item Bab 5 Pengujian dan Eksperimen\\
		Bab 5 memuat hasil pengujian dan eksperimen yang telah dilakukan. Pada bab ini dibahas mengenai skenario pengujian, eksperimen pada algoritma genetika, dan eksperimen pada algoritma \textit{K-means}.
	\item Bab 6 Kesimpulan dan Saran\\
		Bab 6 memuat kesimpulan dari penulis berdasarkan hasil penelitian yang telah dilakukan dan saran untuk peneliti berikutnya agar dapat mengembangkan penelitian ini menjadi lebih baik lagi.
\end{enumerate}