\chapter{Kesimpulan dan Saran}
\label{chap:kesimpulan}

\section{Kesimpulan}
Kesimpulan yang dapat diambil dari penelitian ini adalah sebagai berikut:

\begin{enumerate}
	\item Berdasarkan \textit{dataset} yang telah digunakan, algoritma genetika dapat digunakan dalam pengelompokan dokumen. Namun, diperlukan beberapa adaptasi terhadap komponen-komponen dalam algoritma genetika sebelum dapat digunakan untuk mengelompokkan dokumen. Adaptasi yang perlu dilakukan diantaranya adalah:
	\begin{itemize}
		\item Merepresentasikan dokumen ke dalam suatu model ruang vektor.
		\item Kromosom tersusun atas $K$ \textit{centroid} dalam bentuk vektor.
		\item Fungsi \textit{fitness} yang digunakan adalah menggunakan \textit{intracluster similarity}.
	\end{itemize}
	\item Perangkat lunak yang menggunakan algoritma genetika untuk mengelompokkan dokumen telah berhasil dibuat. Berdasarkan hasil eksperimen menggunakan \textit{dataset} dalam penelitian ini, rata-rata nilai \textit{purity} dari hasil pengelompokan menggunakan algoritma genetika adalah sebesar 0.799, lebih baik 56\% dibandingkan dengan menggunakan algoritma K-means. Hal ini terjadi karena algoritma genetika dapat dengan lebih baik mengatasi \textit{local optimum} dibandingkan dengan algoritma \textit{K-means}. Namun dari segi waktu, algoritma genetika membutuhkan waktu 4365\% lebih lama dibandingkan dengan algoritma \textit{K-means}. Hal ini disebabkan oleh proses komputasi yang dilakukan pada algoritma genetika jauh lebih banyak dan kompleks dibandingkan dengan algoritma \textit{K-means}.
	\item Representasi kromosom yang kurang tepat juga menjadi alasan algoritma genetika berjalan dengan lambat. Mulai generasi kedua, \textit{centroid} yang menyusun kromosom akan bersifat tidak \textit{sparse} (memiliki sedikit elemen bernilai nol). \textit{Centroid} yang tidak \textit{sparse} akan memperlambat proses perhitungan menggunakan \textit{cosine similarity} karena seharusnya perhitungan menggunakan \textit{cosine similarity} dapat mengabaikan elemen berbobot nol. Namun karena banyak elemen yang tidak berbobot nol, maka hanya sedikit elemen yang dapat diabaikan dalam proses perhitungan menggunakan \textit{cosine similarity}.
	\item Metrik yang digunakan dalam penelitian ini yaitu \textit{intracluster similarity} kurang merepresentasikan seberapa baik suatu hasil pengelompokan. Berdasarkan hasil eksperimen, nilai \textit{intracluster} similarity tidak berbanding lurus dengan nilai \textit{purity}. Salah satu kemungkinannya adalah karena semakin jauh jarak suatu anggota \textit{cluster} dari \textit{centroid}, maka nilai \textit{intracluster similarity} akan semakin kecil. Nilai \textit{purity} merupakan suatu nilai biner sehingga sejauh apapun suatu anggota \textit{cluster} dari \textit{centroid}, objek tersebut tetap merupakan anggota dari \textit{cluster}. Kemungkinan yang terjadi adalah banyak objek yang jaraknya cukup jauh dari \textit{centroid} namun tetap merupakan bagian dari \textit{cluster} tersebut karena jarak ke \textit{centroid} lain lebih jauh.
\end{enumerate}

\section{Saran}
Saran dari penulis untuk peneliti selanjutnya agar dapat mengembangkan penelitian ini adalah sebagai berikut:

\begin{enumerate}
	\item Mencari suatu metrik yang lebih mendekati nilai \textit{purity}. Beberapa alternatif metrik yang bisa dicoba untuk mengembangkan penelitian ini adalah dengan menggunakan metrik \textit{intercluster} atau menggunakan metrik \textit{silhouette}.
	\item Menggunakan representasi kromosom yang lain sehingga dapat menjaga agar vektor dari \textit{centroid} tetap \textit{sparse}. Hal ini dapat dilakukan dengan mengubah representasi kromosom menjadi dokumen dan keanggotaannya dalam \textit{cluster}.
\end{enumerate}