\chapter{Perancangan}
\label{chap:perancangan}

Pada bab ini dijelaskan mengenai beberapa perancangan yang dilakukan dalam penelitian ini yaitu rancangan kelas dan rancangan antarmuka pengguna

\section{Kebutuhan Masukan dan Keluaran}
Seluruh kebutuhan masukan dari perangkat lunak ini diakomodasi oleh antarmuka pengguna (Graphical User Interface). Rincian masukan melalui antarmuka pengguna dibahas lebih lanjut pada Subbab \ref{sec:guidesign}. Keluaran dari perangkat lunak ini adalah sebuah \textit{file} berformat \textit{CSV} sehingga dapat dibuka dan diolah lebih lanjut dengan perangkat pengolah \textit{spreadsheet} seperti Microsoft Excel. \textit{File} ini berisi laporan mengenai parameter masukan, hasil pengelompokan, nilai \textit{fitness}, dan waktu yang dibutuhkan untuk melakukan seluruh proses tersebut. Contoh keluaran yang ditulis pada \textit{file} CSV dapat dilihat pada Tabel \ref{tbl:csv-example}. 

Format penamaan dari \textit{file} ini adalah "algoritma-YYYY.MM.DD hh{\_}mm{\_}ss.csv". Sebagai contoh apabila pengelompokan menggunakan algoritma genetika dan dilakukan pada tanggal 30 Januari 2019 pukul 08:15:30 maka nama dari \textit{file} yang dihasilkan adalah "GA-2019.01.30 08{\_}15{\_}30.csv". Apabila algoritma yang digunakan adalah \textit{K-means} maka nama \textit{file} yang dihasilkan adalah "KMeans-2019.01.30 08{\_}15{\_}30.csv".

\begin{table}[H]
\centering
\caption{Contoh keluaran dalam bentuk \textit{file CSV}}
\begin{tabular}{lllll}
Direktori Dokumen:                                 & \multicolumn{4}{l}{D:\textbackslash{}dataset\textbackslash{}bbc} \\
                                                   &                                                                &                                                       &                                                        &                                                       \\
Parameter:                                         &                                                                &                                                       &                                                        &                                                       \\ \cline{1-2}
\multicolumn{1}{|l|}{Banyaknya Cluster}            & \multicolumn{1}{l|}{5}                                         &                                                       &                                                        &                                                       \\ \cline{1-2}
\multicolumn{1}{|l|}{Banyaknya Populasi}           & \multicolumn{1}{l|}{100}                                       &                                                       &                                                        &                                                       \\ \cline{1-2}
\multicolumn{1}{|l|}{Metode Pembobotan}            & \multicolumn{1}{l|}{TF-IDF}                                    &                                                       &                                                        &                                                       \\ \cline{1-2}
\multicolumn{1}{|l|}{Probabilitas Mutasi}          & \multicolumn{1}{l|}{0.05}                                      &                                                       &                                                        &                                                       \\ \cline{1-2}
\multicolumn{1}{|l|}{Maksimum Iterasi}             & \multicolumn{1}{l|}{100}                                       &                                                       &                                                        &                                                       \\ \cline{1-2}
\multicolumn{1}{|l|}{Individu Elitisme}            & \multicolumn{1}{l|}{1}                                         &                                                       &                                                        &                                                       \\ \cline{1-2}
\multicolumn{1}{|l|}{Banyaknya Generasi Konvergen} & \multicolumn{1}{l|}{3}                                         &                                                       &                                                        &                                                       \\ \cline{1-2}
\multicolumn{1}{|l|}{Batas Konvergen}              & \multicolumn{1}{l|}{1.00E-05}                                  &                                                       &                                                        &                                                       \\ \cline{1-2}
                                                   &                                                                &                                                       &                                                        &                                                       \\
Hasil:                                             &                                                                &                                                       &                                                        &                                                       \\ \cline{1-2}
\multicolumn{1}{|l|}{Waktu}                        & \multicolumn{1}{l|}{20 menit 6 detik}                          &                                                       &                                                        &                                                       \\ \cline{1-2}
\multicolumn{1}{|l|}{Intracluster}                 & \multicolumn{1}{l|}{565.9675341}                               &                                                       &                                                        &                                                       \\ \cline{1-2}
\multicolumn{1}{|l|}{Banyak Iterasi}               & \multicolumn{1}{l|}{4}                                         &                                                       &                                                        &                                                       \\ \cline{1-2}
                                                   &                                                                &                                                       &                                                        &                                                       \\
Hasil Clustering:                                  &                                                                &                                                       &                                                        &                                                       \\
C1                                                 & C2                                                             & C3                                                    & C4                                                     & C5                                                    \\ \hline
\multicolumn{1}{|l|}{tech\textbackslash{}020.txt}  & \multicolumn{1}{l|}{entertainment\textbackslash{}267.txt}      & \multicolumn{1}{l|}{tech\textbackslash{}207.txt}      & \multicolumn{1}{l|}{sport\textbackslash{}418.txt}      & \multicolumn{1}{l|}{sport\textbackslash{}262.txt}     \\ \hline
\multicolumn{1}{|l|}{tech\textbackslash{}128.txt}  & \multicolumn{1}{l|}{entertainment\textbackslash{}279.txt}      & \multicolumn{1}{l|}{tech\textbackslash{}234.txt}      & \multicolumn{1}{l|}{sport\textbackslash{}134.txt}      & \multicolumn{1}{l|}{sport\textbackslash{}391.txt}     \\ \hline
\multicolumn{1}{|l|}{tech\textbackslash{}012.txt}  & \multicolumn{1}{l|}{business\textbackslash{}220.txt}           & \multicolumn{1}{l|}{tech\textbackslash{}229.txt}      & \multicolumn{1}{l|}{sport\textbackslash{}394.txt}      & \multicolumn{1}{l|}{sport\textbackslash{}389.txt}     \\ \hline
\multicolumn{1}{|l|}{tech\textbackslash{}014.txt}  & \multicolumn{1}{l|}{entertainment\textbackslash{}342.txt}      & \multicolumn{1}{l|}{tech\textbackslash{}005.txt}      & \multicolumn{1}{l|}{sport\textbackslash{}379.txt}      & \multicolumn{1}{l|}{sport\textbackslash{}243.txt}     \\ \hline
\multicolumn{1}{|l|}{tech\textbackslash{}334.txt}  & \multicolumn{1}{l|}{sport\textbackslash{}471.txt}              & \multicolumn{1}{l|}{tech\textbackslash{}359.txt}      & \multicolumn{1}{l|}{sport\textbackslash{}305.txt}      & \multicolumn{1}{l|}{sport\textbackslash{}206.txt}     \\ \hline
\multicolumn{1}{|c|}{...}                          & \multicolumn{1}{c|}{...}                                       & \multicolumn{1}{c|}{...}                              & \multicolumn{1}{c|}{...}                               & \multicolumn{1}{c|}{...}                             
\end{tabular}
\label{tbl:csv-example}
\end{table}

\newpage
\section{Rancangan Kelas}
Berdasarkan hasil analisis dari masalah yang dihadapi, dibentuklah diagram kelas pada Gambar \ref{fig:diagramkelas} sebagai gambaran dari perangkat lunak yang dibuat.

\begin{figure}[H]
	\begin{center}
		\includegraphics[width=\textwidth]{DiagramKelas/Full}
		\caption{Diagram kelas}
		\label{fig:diagramkelas}
	\end{center}
\end{figure}

Gambar \ref{fig:diagramkelas} merupakan diagram kelas secara umum yang tidak memuat atribut dan \textit{method} dari setiap kelas. Diagram kelas ini sengaja dibuat agar hubungan antar kelas dapat dengan lebih mudah dilihat. Diagram kelas pada Gambar \ref{fig:diagramkelas} dibagi ke dalam tiga \textit{package} berdasarkan fungsinya. \textit{Package} IR berfungsi untuk mengambil informasi dari dokumen input. \textit{Package} GA berfungsi untuk melakukan pengelompokan dokumen dengan menggunakan algoritma genetika. Sedangkan \textit{package} \textit{K-means} berfungsi untuk melakukan pengelompokan dokumen menggunakan algoritma \textit{K-means}. Beberapa kelas dalam penelitian ini dibuat sebagai kelas \textit{singleton} untuk menjaga integritas data yang disimpan saat ada lebih dari satu \textit{instance} yang ingin mengakses atribut-atributnya. Penjelasan dari setiap kelas dalam Gambar \ref{fig:diagramkelas} adalah sebagai berikut.

\subsection{\textit{Document}}

\begin{figure}[H]
	\begin{center}
		\includegraphics[width=0.4\textwidth]{DiagramKelas/Document}
		\caption{Kelas \textit{Document}}
		\label{fig:kelasDocument}
	\end{center}
\end{figure}

Kelas ini merupakan representasi dari dokumen yang diproses dalam pengelompokan. Kelas ini berfungsi untuk menyimpan informasi yang dibutuhkan dari sebuah dokumen selama proses pengelompokan. Atribut yang dimiliki oleh kelas \textit{Document} adalah:

\begin{itemize}
	\item \textit{file}: atribut ini bertipe \textit{File} milik \textit{package} java.io yang berfungsi untuk merepresentasikan \textit{file} dari dokumen yang diproses.
	\item \textit{wordCount}: atribut ini bertipe \textit{HashMap} dengan \textit{key} bertipe \textit{String} dan \textit{value} bertipe \textit{Integer}. Atribut ini menyimpan pasangan kata yang dimiliki oleh dokumen tersebut dan frekuensinya.
	\item \textit{vector}: atribut bertipe \textit{Vector} ini merepresentasikan model ruang vektor pada sebuah dokumen.
\end{itemize}

\textit{Method} yang terdapat dalam kelas ini adalah:

\begin{itemize}
	\item \textit{Document}: \textit{method} ini merupakan \textit{constructor} dengan sebuah parameter bertipe \textit{File} yaitu \textit{file} dari dokumen yang dikelompokkan.
	\item \textit{indexDocument}: \textit{method} tanpa kembalian (\textit{void}) yang berfungsi untuk mengindeks dokumen untuk mengisi atribut \textit{wordCount}.
	\item \textit{getWordCount}: \textit{method} yang berfungsi untuk mengembalikan banyaknya \textit{term} muncul dalam dokumen.
	\item \textit{getVector}: \textit{method} ini merupakan \textit{getter} dari atribut \textit{vector}.
	\item \textit{getDocName}: \textit{method} ini mengembalikan nama \textit{file} dari dokumen.
\end{itemize}

\subsection{\textit{Vector}}

\begin{figure}[H]
	\begin{center}
		\includegraphics[width=0.7\textwidth]{DiagramKelas/Vector}
		\caption{Kelas \textit{Vector}}
		\label{fig:kelasVector}
	\end{center}
\end{figure}

Kelas ini merupakan kelas yang merepresentasikan sebuah vektor. Kelas ini memiliki fungsi dan atribut yang berfungsi untuk menunjang seluruh aktivitas yang melibatkan suatu vektor. Atribut yang dimiliki kelas ini adalah:

\begin{itemize}
	\item \textit{termsWeight}: atribut ini bertipe \textit{Hashmap} dengan \textit{key} berupa \textit{String} dan \textit{value} berupa \textit{Double}. Atribut ini menyimpan pasangan \term dan bobotnya sesuai dengan metode pembobotan.
	\item \textit{similarityCalculator}: atribut ini bertipe \textit{SimilarityCalculator} dan merupakan objek yang digunakan untuk menghitung kemiripan antar vektor.
	\item \textit{termWeighting}: atribut ini bertipe \textit{TermWeighting} dan merupakan objek yang digunakan untuk menghitung bobot dari setiap dimensi dalam suatu vektor.
\end{itemize}

\textit{Method} yang terdapat dalam kelas ini adalah:

\begin{itemize}
	\item \textit{Vector}: \textit{method} ini merupakan constructor dengan sebuah parameter yaitu \textit{wordCount} bertipe \textit{Hashmap<String,Integer>} yang merupakan pasangan kata dan banyak kemunculannya dalam dokumen.
	\item \textit{Vector}: \textit{method} ini merupakan constructor dengan sebuah parameter yaitu objek \textit{Vector} yang berfungsi untuk menduplikasi objek \textit{Vector}.
	\item \textit{generateVector}: \textit{method} ini merupakan \textit{method} dengan sebuah parameter \textit{Hashmap<String,Integer>} untuk mengisi atribut \textit{termsWeight} dengan bobot tiap \term.
	\item \textit{getWeight}: \textit{method} ini berfungsi untuk mengembalikan bobot dari \term.
	\item \textit{calculateWeight}: \textit{method} ini membutuhkan dua buah parameter yaitu sebuah \term bertipe \textit{String} dan sebuah \textit{Hashmap<String,Integer>}. \textit{Method} ini berfungsi untuk menghitung bobot dari \term berdasarkan frekuensi yang terdapat pada \textit{HashMap}.
	\item \textit{calculateSimilarity}: \textit{method} ini berfungsi untuk menghitung kemiripan (\textit{similarity}) antara vektor ini dengan \textit{otherVector} menggunakan metode yang dipilih oleh pengguna.
	\item \textit{mutate}: \textit{method} ini berfungsi untuk melakukan operasi mutasi pada tingkat vektor.
	\item \textit{setTermsWeight}: \textit{method} ini merupakan \textit{setter} dari atribut \textit{termsWeight}.
	\item \textit{setWeight}: \textit{method} ini berfungsi untuk mengubah bobot \term menjadi \textit{value}.
	\item \textit{getLength}: \textit{method} ini berfungsi untuk mendapatkan panjang dari vektor.
	\item \textit{getKeyset}: \textit{method} ini berfungsi untuk mendapatkan himpunan \term yang ada pada vektor ini.
	\item \textit{getTermsWeight}: \textit{method} ini merupakan \textit{getter} dari atribut \textit{termsWeight}.
	\item \textit{getDimension}: \textit{method} ini berfungsi untuk mendapatkan besarnya dimensi dari vektor ini.
\end{itemize}

\subsection{\textit{SimilarityCalculator}}

\begin{figure}[H]
	\begin{center}
		\includegraphics[width=0.6\textwidth]{DiagramKelas/SimilarityCalculator}
		\caption{Kelas \textit{SimilarityCalculator}}
		\label{fig:kelasSimilarityCalculator}
	\end{center}
\end{figure}

Kelas ini merupakan kelas abstrak yang berfungsi untuk menghitung kemiripan antara dua buah objek bertipe \textit{Vector}. Kelas ini tidak memiliki atribut dan hanya memiliki sebuah \textit{method} abstrak yaitu \textit{calculateSimilarity} yang memiliki dua buah parameter \textit{vector1} dan \textit{vector2}. Hasil yang dikembalikan oleh \textit{method} ini adalah kemiripan dari kedua vektor tersebut sesuai dengan metode perhitungan jaraknya.

\subsection{\textit{CosineSimilarityCalculator}}

\begin{figure}[H]
	\begin{center}
		\includegraphics[width=0.6\textwidth]{DiagramKelas/CosineSimilarityCalculator}
		\caption{Kelas \textit{CosineSimilarityCalculator}}
		\label{fig:kelasCosineDist}
	\end{center}
\end{figure}

Kelas ini mengimplementasikan kelas abstrak \textit{SimilarityCalculator}. Kelas ini memiliki satu \textit{method} tambahan selain melakukan \textit{override} pada \textit{method calculateSimilarity}. \textit{Method} yang ada pada kelas ini adalah:

\begin{itemize}
	\item \textit{calculateDistance}: \textit{method} ini merupakan \textit{method} yang diturunkan dari kelas \textit{SimilarityCalculator}. \textit{Method} ini mengembalikan \textit{similarity} dari \textit{vector1} dan \textit{vector2} yang dihitung menggunakan persamaan cosinus (Persamaan \ref{eq:cosine}).
	\item \textit{dotProduct}: \textit{method} ini berfungsi untuk menghitung hasil perkalian titik (\textit{dot product}) antara \textit{vector1} dan \textit{vector2}.
\end{itemize}

\subsection{\textit{TermWeighting}}

\begin{figure}[H]
	\begin{center}
		\includegraphics[width=0.6\textwidth]{DiagramKelas/TermWeighting}
		\caption{Kelas \textit{TermWeighting}}
		\label{fig:kelasTermWeighting}
	\end{center}
\end{figure}

Kelas ini merupakan kelas abstrak yang merepresentasikan metode perhitungan bobot dalam suatu vektor. Kelas ini hanya memiliki satu buah \textit{method} yaitu \textit{calculateWeight}. \textit{Method} ini berfungsi untuk menghitung bobot dari \term berdasarkan metode pembobotan yang dipilih oleh pengguna.

\subsection{\textit{FrequencyWeighting}}

\begin{figure}[H]
	\begin{center}
		\includegraphics[width=0.6\textwidth]{DiagramKelas/FrequencyWeighting}
		\caption{Kelas \textit{FrequencyWeighting}}
		\label{fig:kelasFrequencyWeighting}
	\end{center}
\end{figure}

Kelas ini mengimplementasikan kelas abstrak \textit{TermWeighting}. Kelas ini hanya memiliki satu \textit{method} yang diturunkan langsung dari kelas \textit{TermWeighting} yaitu \textit{method calculateWeight}. \textit{Method} ini berfungsi untuk menghitung bobot dari \textit{term} menggunakan bobot frekuensi seperti yang telah dijelaskan pada Subbab \ref{sub:freq}.

\subsection{\textit{TFIDFWeighting}}

\begin{figure}[H]
	\begin{center}
		\includegraphics[width=0.6\textwidth]{DiagramKelas/TFIDFWeighting}
		\caption{Kelas \textit{TFIDFWeighting}}
		\label{fig:kelasTFIDFWeighting}
	\end{center}
\end{figure}

Kelas ini mengimplementasikan kelas abstrak \textit{TermWeighting}. Kelas ini memiliki dua \textit{method} tambahan selain melakukan \textit{override} pada \textit{method calculateWeight}. \textit{Method} yang ada pada kelas ini adalah:

\begin{itemize}
	\item \textit{calculateWeight}: \textit{method} ini merupakan \textit{method} yang diturunkan dari kelas \textit{TermWeighting}. \textit{Method} ini mengembalikan bobot dari \term yang dihitung menggunakan teknik TF-IDF.
	\item \textit{calculateTF}: \textit{method} ini berfungsi untuk menghitung \textit{TF} dari suatu \textit{term}.
	\item \textit{calculateIDF}: \textit{method} ini berfungsi untuk menghitung \textit{IDF} dari suatu \textit{term}.
\end{itemize}

\subsection{\textit{Lexicon}}

\begin{figure}[H]
	\begin{center}
		\includegraphics[width=0.55\textwidth]{DiagramKelas/Lexicon}
		\caption{Kelas \textit{Lexicon}}
		\label{fig:kelasLexicon}
	\end{center}
\end{figure}

Kelas ini merepresentasikan sebuah kamus yang menangani seluruh kebutuhan dalam proses pengelompokan yang membutuhkan akses global untuk keseluruhan koleksi dokumen. Atribut yang ada dalam kelas ini adalah:

\begin{itemize}
	\item \textit{globalTermCount}: atribut ini bertipe \textit{HashMap} dengan \textit{key} bertipe \textit{String} dan \textit{value} bertipe \textit{Integer}. Atribut ini berfungsi untuk menyimpan seluruh \term yang muncul dan banyak kemunculannya dalam keseluruhan koleksi dokumen.
	\item \textit{documentFrequency}: atribut ini bertipe \textit{HashMap} dengan \textit{key} bertipe \textit{String} dan \textit{value} bertipe \textit{Integer}. Atribut ini berfungsi untuk menyimpan seluruh frekuensi dokumen dari tiap term.
	\item \textit{instance}: atribut ini merupakan objek bertipe \textit{Lexicon} sebagai instansiasi satu-satunya dari kelas \textit{Lexicon} karena kelas ini bersifat \textit{singleton}.
	\item \textit{numberOfDocument}: atribut ini menyimpan banyaknya dokumen yang terdaftar di \textit{Lexicon}. 
\end{itemize}

\textit{Method} yang ada pada kelas ini adalah:

\begin{itemize}
	\item \textit{Lexicon}: \textit{method} ini merupakan \textit{constructor private} untuk menjamin tidak akan ada lebih dari satu \textit{instance} selama perangkat lunak berjalan.
	\item \textit{getInstance}: \textit{method} ini merupakan \textit{method static} yang berfungsi sebagai \textit{getter} dari atribut \textit{instance}.
	\item \textit{insertTerm}: \textit{method} ini berfungsi untuk memasukkan \textit{term} ke dalam atribut \textit{globalTermCount}.
	\item \textit{getAllTermList}: \textit{method} ini bertugas mengembalikan daftar seluruh \term yang pernah muncul di seluruh koleksi dokumen.
	\item \textit{updateDF}: \textit{method} ini berfungsi untuk menambah nilai TF dari \term "\textit{term}".
	\item \textit{getNumberOfDocument}: \textit{method} ini merupakan \textit{getter} dari atribut \textit{numberOfDocument}.
	\item \textit{setNumberOfDocument}: \textit{method} ini merupakan \textit{setter} dari atribut \textit{numberOfDocument}.
	\item \textit{getDocumentFrequency}: \textit{method} ini berfungsi untuk mendapatkan nilai DF dari \term "\textit{term}".
\end{itemize}

\subsection{\textit{Gene}}

\begin{figure}[H]
	\begin{center}
		\includegraphics[width=0.25\textwidth]{DiagramKelas/Gene}
		\caption{Kelas \textit{Gene}}
		\label{fig:kelasGene}
	\end{center}
\end{figure}

Kelas ini merepresentasikan gen dalam algoritma genetika. Kelas ini hanya memiliki sebuah atribut \textit{value} bertipe \textit{Vector}. Atribut ini menyimpan vektor yang menjadi titik pusat \textit{cluster} (\textit{centroid}). \textit{Method} yang ada pada kelas ini adalah:

\begin{itemize}
	\item \textit{Gene}: \textit{method} ini merupakan \textit{constructor} dari kelas \textit{Gene} yang membutuhkan sebuah parameter bertipe \textit{Vector} untuk mengisi atribut \textit{value}.
	\item \textit{Gene}: \textit{method} ini merupakan \textit{overloading constructor} yang berfungsi untuk menduplikasi objek dari kelas \textit{Gene}.
	\item \textit{getValue}: \textit{method} ini merupakan \textit{getter} dari atribut \textit{value}.
	\item \textit{mutate}: \textit{method} ini berfungsi untuk melakukan mutasi pada gen. \textit{Method} ini sebenarnya hanya bertugas memanggil fungsi \textit{mutate()} dari atribut \textit{value}.
\end{itemize}

\subsection{\textit{Chromosome}}

\begin{figure}[H]
	\begin{center}
		\includegraphics[width=0.6\textwidth]{DiagramKelas/Chromosome}
		\caption{Kelas \textit{Chromosome}}
		\label{fig:kelasChromosome}
	\end{center}
\end{figure}

Kelas ini merepresentasikan kromosom dalam algoritma genetika. Atribut yang terdapat dalam kelas ini adalah:

\begin{itemize}
	\item \textit{genes}: atribut bertipe \textit{List of Gene} dan merupakan kumpulan gen yang terdapat dalam kromosom.
	\item \textit{rand}: atribut ini merupakan objek \textit{Random} milik \textit{Java} dan berfungsi untuk membangkitkan bilangan acak yang dibutuhkan dalam setiap proses dalam kromosom.
	\item \textit{fitnessValue}: atribut ini menyimpan nilai \textit{fitness} dari kromosom.
	\item \textit{clusteringResult}: atribut ini menyimpan hasil dari pengelompokan. Atribut ini bertipe \textit{HashMap} yang menyimpan pasangan dokumen dan \textit{cluster} dari dokumen tersebut.
\end{itemize}

\textit{Method} yang terdapat dalam kelas ini adalah:

\begin{itemize}
	\item \textit{Chromosome}: \textit{method} ini merupakan \textit{constructor} tanpa parameter untuk membentuk objek dari kelas \textit{Chromosome}.
	\item \textit{addGene}: \textit{method} ini bertugas untuk menambahkan satu gen ke dalam kromosom (ke dalam atribut \textit{genes}).
	\item \textit{mutate}: \textit{method} ini berfungsi untuk melakukan mutasi pada kromosom dengan cara melakukan mutasi pada sebuah gen secara acak (Subbab \ref{sub:mutation}).
	\item \textit{crossover}: \textit{method} ini bertugas untuk melakukan persilangan dengan kromosom \textit{otherChromosome} untuk mengahsilkan keturunan (Subbab \ref{sub:crossover}).
	\item \textit{determineCluster}: \textit{method} ini berfungsi untuk menentukan keanggotaan dari setiap dokumen.
	\item \textit{computeFitness}: \textit{method} ini mengembalikan nilai \textit{fitness} dari kromosom (Subbab \ref{sub:fitness}).
	\item \textit{getFitness}: \textit{method} ini merupakan \textit{getter} dari atribut \textit{fitnessValue}.
	\item \textit{getAllGenes}: \textit{method} ini merupakan \textit{getter} dari atribut \textit{genes}.
	\item \textit{getClusteringResult}: \textit{method} ini merupakan \textit{getter} dari atribut \textit{clusteringResult}.
	\item \textit{compareTo}: \textit{method} ini merupakan turunan dari \textit{interface Comparable} milik \textit{Java} yang dibutuhkan untuk membandingkan kromosom ini dengan kromosom \textit{o}. Method ini nantinya digunakan dalam proses \textit{sorting}.
\end{itemize}

\subsection{\textit{GAClusterer}}

\begin{figure}[H]
	\begin{center}
		\includegraphics[width=0.7\textwidth]{DiagramKelas/GAClusterer}
		\caption{Kelas \textit{GAClusterer}}
		\label{fig:kelasGAClusterer}
	\end{center}
\end{figure}

Kelas ini merupakan kelas utama yang mengatur jalannya proses pengelompokan menggunakan algoritma genetika. Kelas ini merupakan kelas \textit{singleton}. Atribut yang terdapat dalam kelas ini adalah:

\begin{itemize}
	\item \textit{population}: atribut ini bertipe \textit{List of Chromosome} yang merepresentasikan populasi pada generasi saat ini.
	\item \textit{docs}: atribut ini bertipe \textit{List of Document} yang berfungsi untuk menyimpan seluruh koleksi dokumen.
	\item \textit{instance}: atribut \textit{static} ini berfungsi untuk menyimpan \textit{instance} dari kelas \textit{GAClusterer}.
	\item \textit{solutionList}: atribut ini bertipe \textit{List of Chromosome} yang mencatat kromosom dengan nilai \textit{fitness} terbaik untuk setiap generasinya.
	\item \textit{isRunning}: atribut ini berfungsi untuk menyimpan status dari operasi pengelompokan menggunakan GA. Apabila bernilai \textit{true} maka program sedang berjalan dan \textit{false} apabila tidak sedang berjalan.
	\item \textit{progress}: atribut ini bertipe \textit{ReadOnlyDoubleWrapper} dan berfungsi untuk menyimpan perkembangan dari pengerjaan tugas pengelompokan ini ke antarmuka pengguna.
\end{itemize}

\textit{Method} yang terdapat dalam kelas ini adalah:

\begin{itemize}
	\item \textit{GAClusterer}: \textit{method} ini merupakan \textit{constructor private} yang berfungsi untuk menjamin tidak akan ada \textit{instance} dibuat diluar dari kelas ini.
	\item \textit{setIsRunning}: \textit{method} ini mrupakan \textit{setter} dari atribut \textit{isRunning}.
	\item \textit{getProgress}: \textit{method} ini mengembalikan perkembangan pekerjaan program (skala 0 sampai dengan 1).
	\item \textit{progressProperty}: \textit{method} ini merupakan \textit{getter} dari atribut \textit{progress}.
	\item \textit{getInstance}: \textit{method} ini merupakan \textit{getter} dari atribut \textit{instance}.
	\item \textit{rouletteWheelSelect}: \textit{method} ini bertugas untuk memilih sebuah kromosom menggunakan teknik \textit{roulette-wheel selection} dari populasi.
	\item \textit{elitism}: \textit{method} ini bertugas untuk memilih $n$ kromosom elit yang langsung masuk ke generasi berikutnya. Cara kerja \textit{method} ini dijelaskan dalam Algoritma \ref{alg:elitism}.
	
	\begin{algorithm}[H]
	%fungsi persilangan
	\caption{Elitisme Algoritma Genetika}
	\label{alg:elitism}
	\begin{flushleft}
		\textbf{function} ELITISME($count$, $populasi$) \textbf{returns} individu-individu elit
		\begin{flushleft}
			\begin{tabular}{ l l }
				\textbf{inputs:}& $count$, banyaknya individu elit\\
				& $populasi$, himpunan individu
				\hspace{5pt} 
			\end{tabular} 
		\end{flushleft}
	\end{flushleft}

	\begin{algorithmic}[1]
		\STATE $pq \leftarrow$ \textit{Priority Queue}
		\FOR{$i$=1 \textbf{to} SIZE($populasi$)} \label{alg:elitism:ln-2}
			\IF{$count == 0$ } \label{alg:elitism:ln-3}
				\STATE break \label{alg:elitism:ln-4}
			\ENDIF
			\IF{$i < count$ } \label{alg:elitism:ln-6}
				\STATE OFFER($pq$, $populasi[i]$) \label{alg:elitism:ln-7}
			\ELSE
				\IF{FITNESS($populasi[i]$) > FITNESS(PEEK($pq$))} \label{alg:elitism:ln-9}
					\STATE POLL($pq$) \label{alg:elitism:ln-10}
					\STATE OFFER($pq$,$populasi[i]$) \label{alg:elitism:ln-11}
				\ENDIF
			\ENDIF
		\ENDFOR
		\STATE \textbf{return} TO-ARRAY($pq$) \label{alg:elitism:ln-15}
	\end{algorithmic}
\end{algorithm}

\pagebreak
	Penjelasan untuk fungsi  Elitisme adalah sebagai berikut:
	\begin{itemize}
		\item Individu elit akan dipilih dengan cara mengiterasi seluruh $populasi$ (baris \ref{alg:elitism:ln-2})
		\item Apabila jumlah elit yang diinginkan adalah 0 ($count ==0$), maka iterasi akan diberhentikan (baris \ref{alg:elitism:ln-3} dan \ref{alg:elitism:ln-4}).
		\item Apabila banyaknya individu dalam $pq$ masih kurang dari $count$ (baris \ref{alg:elitism:ln-6}), maka akan dilakukan pemanggilan fungsi \textit{OFFER} yang berfungsi untuk memasukkan individu ke-$i$ dari populasi ke dalam $pq$ (baris \ref{alg:elitism:ln-7}).
		\item Pada baris \ref{alg:elitism:ln-9}, fungsi \textit{PEEK} berfungsi untuk mengembalikan elemen paling awal dalam $pq$. Kemudian akan dibandingkan nilai \textit{fitness} antara elemen paling awal $pq$ dan individu ke-$i$ dalam populasi.
		\item Apabila nilai \textit{fitness} individu ke-$i$ dalam populasi lebih besar daripada elemen paling awal dalam $pq$, maka akan dilakukan dua hal dalam baris \ref{alg:elitism:ln-10} dan \ref{alg:elitism:ln-11}.\
		\item Pada baris \ref{alg:elitism:ln-10}, fungsi \textit{POLL} berfungsi untuk mengeluarkan elemen paling awal dalam $pq$.
		\item Pada baris \ref{alg:elitism:ln-11}, fungsi \textit{OFFER} akan memasukkan individu ke-$i$ dari populasi ke dalam $pq$.
		\item Pada baris \ref{alg:elitism:ln-15}, fungsi \textit{TO-ARRAY} akan mengubah $pq$ ke dalam bentuk \textit{array}.
	\end{itemize}
	
	\item \textit{selection}: \textit{method} ini bertugas untuk melakukan \textit{roulette-wheel selection} sebanyak populasi untuk menghasilkan populasi dari generasi selanjutnya.
	\item \textit{initialize}: \textit{method} ini berfungsi untuk melakukan \textit{indexing} dokumen dan membentuk populasi awal.
	\item \textit{getAllDocs}: \textit{method} ini merupakan \textit{getter} dari atribut \textit{docs}.
	\item \textit{cluster}: \textit{method} ini merupakan \textit{method} utama yang bertugas melakukan pengelompokan dokumen dengan menggunakan algoritma genetika.
	\item \textit{getSolution}: \textit{method} ini berfungsi untuk menggembalikan solusi dari proses pengelompokan menggunakan algoritma genetika.
	\item \textit{reset}: \textit{method} ini berfungsi untuk mengatur ulang seluruh atribut untuk proses pengelompokan berikutnya.
\end{itemize}

\subsection{\textit{Params}}

\begin{figure}[H]
	\begin{center}
		\includegraphics[width=\textwidth]{DiagramKelas/Params}
		\caption{Kelas \textit{Params}}
		\label{fig:kelasParams}
	\end{center}
\end{figure}

Kelas ini berfungsi untuk menyimpan seluruh parameter yang diberikan oleh pengguna agar dapat digunakan oleh setiap kelas yang membutuhkannya. Kelas ini bersifat \textit{singleton} sehingga hanya ada satu buah \textit{instance} selama perangkat lunak berjalan. Atribut yang ada dalam kelas ini adalah:

\begin{itemize}
	\item \textit{instance}: atribut ini merupakan objek bertipe \textit{Params} sebagai instansiasi satu-satunya dari kelas \textit{Params} karena kelas ini bersifat \textit{singleton}.
	\item \textit{filepath}: atribut ini berfungsi untuk menyimpan alamat dari direktori dokumen yang dikelompokkan.
	\item \textit{K}: atribut ini berfungsi untuk menyimpan banyaknya \textit{cluster} yang dipentuk dalam proses pengelompokan.
	\item \textit{P}: atribut ini berfungsi untuk menyimpan banyaknya populasi yang dibentuk dalam proses pengelompokan menggunakan algoritma genetika.
	\item \textit{weightingMethod}: atribut ini berfungsi untuk menyimpan metode pembobotan yang digunakan dalam proses pengelompokan. Apabila atribut ini bernilai 0 maka metode yang digunakan adalah TF-IDF sedangkan apabila bernilai 1 maka metode yang digunakan adalah bobot frekuensi.
	\item \textit{mu}\_\textit{m}: atribut ini berfungsi untuk menyimpan probabilitas mutasi dalam bentuk bilangan riil bernilai antara 0 sampai dengan 1.
	\item \textit{maxIt}: atribut ini berfungsi untuk menyimpan banyaknya iterasi maksimal yang dapat dilakukan.
	\item \textit{elitismCount}: atribut ini berfungsi untuk menyimpan banyaknya individu yang dijadikan elit pada tahap seleksi.
	\item \textit{convergeGen}: atribut ini berfungsi untuk menyimpan banyaknya generasi konvergen sebelum proses pengelompokan diberhentikan.
	\item \textit{convergeEpsilon}: atribut ini berfungsi untuk menyimpan nilai yang digunakan untuk membandingkan \textit{fitness} tiap solusi dari setiap generasi untuk menentukan apakah sudah tercapai konvergen atau belum.
\end{itemize}

\textit{Method} yang ada pada kelas ini adalah:

\begin{itemize}
	\item \textit{Params}: \textit{method} ini merupakan \textit{constructor private} untuk menjamin tidak akan ada lebih dari satu \textit{instance} selama perangkat lunak berjalan.
	\item \textit{getInstance}: \textit{method} ini merupakan \textit{method static} yang berfungsi sebagai \textit{getter} dari atribut \textit{instance}.
	\item \textit{insertParam}: \textit{method} ini bertugas untuk memasukkan atau mengubah nilai dari setiap atribut dalam kelas ini.
	\item \textit{getK}: \textit{method} ini merupakan \textit{getter} dari atribut \textit{K}.
	\item \textit{getP}: \textit{method} ini merupakan \textit{getter} dari atribut \textit{P}.
	\item \textit{getWeightingMethod}: \textit{method} ini merupakan \textit{getter} dari atribut \textit{weightingMethod}.
	\item \textit{getMu}\_\textit{m}: \textit{method} ini merupakan \textit{getter} dari atribut \textit{mu}\_\textit{m}.
	\item \textit{getMaxIt}: \textit{method} ini merupakan \textit{getter} dari atribut \textit{maxIt}.
	\item \textit{getElitismCount}: \textit{method} ini merupakan \textit{getter} dari atribut \textit{elitismCount}.
	\item \textit{getConvergeGen}: \textit{method} ini merupakan \textit{getter} dari atribut \textit{convergeGen}.
	\item \textit{getConvergeEpsilon}: \textit{method} ini merupakan \textit{getter} dari atribut \textit{convergeEpsilon}.
	\item \textit{getFilepath}: \textit{method} ini merupakan \textit{getter} dari atribut \textit{filepath}.
\end{itemize}.

\subsection{\textit{KMeans}}

\begin{figure}[H]
	\begin{center}
		\includegraphics[width=0.75\textwidth]{DiagramKelas/KMeans}
		\caption{Kelas \textit{KMeans}}
		\label{fig:kelasKMeans}
	\end{center}
\end{figure}

Kelas ini merupakan kelas utama yang mengatur jalannya proses pengelompokan menggunakan algoritma \textit{K-means}. Kelas ini merupakan kelas \textit{singleton}. Atribut yang terdapat dalam kelas ini adalah:

\begin{itemize}
	\item \textit{docs}: atribut ini bertipe \textit{List of Document} yang berfungsi untuk menyimpan seluruh koleksi dokumen.
	\item \textit{solution}: atribut ini berfungsi untuk menyimpan hasil pengelompokan dalam bentuk himpunan pasangan dokumen dan keanggotaan \textit{cluster} dari dokumen tersebut.
	\item \textit{instance}: atribut \textit{static} ini berfungsi untuk menyimpan \textit{instance} dari kelas \textit{KMeans}.
	\item \textit{solutionIntracluster}: atribut ini berfungsi untuk menyimpan nilai \textit{intracluster} dari solusi yang telah didapatkan dari proses pengelompokan.
	\item \textit{isRunning}: atribut ini berfungsi untuk menyimpan status dari operasi pengelompokan menggunakan GA. Apabila bernilai \textit{true} maka program sedang berjalan dan \textit{false} apabila tidak sedang berjalan.
	\item \textit{progress}: atribut ini bertipe \textit{ReadOnlyDoubleWrapper} dan berfungsi untuk menyimpan perkembangan dari pengerjaan tugas pengelompokan ini ke antarmuka pengguna.
\end{itemize}

\textit{Method} yang terdapat dalam kelas ini adalah:

\begin{itemize}
	\item \textit{KMeans}: \textit{method} ini merupakan \textit{constructor private} yang berfungsi untuk menjamin tidak ada \textit{instance} dibuat diluar dari kelas ini.
	\item \textit{setIsRunning}: \textit{method} ini mrupakan \textit{setter} dari atribut \textit{isRunning}.
	\item \textit{getProgress}: \textit{method} ini mengembalikan perkembangan pekerjaan program (skala 0 sampai dengan 1).
	\item \textit{progressProperty}: \textit{method} ini merupakan \textit{getter} dari atribut \textit{progress}.
	\item \textit{getSolution}: \textit{method} ini merupakan \textit{getter} dari atribut \textit{solution}.
	\item \textit{getInstance}: \textit{method} ini merupakan \textit{getter} dari atribut \textit{instance}.
	\item \textit{cluster}: \textit{method} ini merupakan \textit{method} utama yang bertugas melakukan pengelompokan dokumen dengan menggunakan algoritma \textit{K-means}.
	\item \textit{getSolutionIntracluster}: \textit{method} ini merupakan \textit{getter} dari atribut \textit{solutionIntracluster}.
	\item \textit{computeIntracluster}: \textit{method} ini bertugas untuk menghitung \textit{intracluster} apabila diketahui \textit{centroid} setiap \textit{cluster} adalah \textit{centroids} dan keanggotaan setiap dokumen adalah \textit{cluster}.
	\item \textit{determineCluster}: \textit{method} ini berfungsi untuk menentukan keanggotaan dari setiap dokumen.
	\item \textit{reset}: \textit{method} ini berfungsi untuk mengatur ulang seluruh atribut untuk proses pengelompokan berikutnya.
\end{itemize}

\subsection{\textit{FXMLDocumentController}}

\begin{figure}[H]
	\begin{center}
		\includegraphics[width=0.75\textwidth]{DiagramKelas/FXMLDocumentController}
		\caption{Kelas \textit{FXMLDocumentController}}
		\label{fig:kelasFXMLDocumentController}
	\end{center}
\end{figure}

Kelas ini merupakan kelas yang bertugas untuk mengendalikan seluruh aktivitas yang ada di antarmuka dan menghubungkannya dengan kelas lain yang dibutuhkan. Atribut yang terdapat dalam kelas ini adalah:

\begin{itemize}
	\item \textit{textFieldDokumen}: atribut ini bertipe \textit{TextField} dan berfungsi menampilkan direktori dokumen pada antarmuka pengguna.
	\item \textit{buttonDokumen}: atribut ini bertipe \textit{Button} yang merupakan tombol untuk memilih direktori dokumen pada antarmuka pengguna.
	\item \textit{spinnerCluster}: atribut ini bertipe \textit{Spinner} yang menangani masukan untuk parameter banyaknya \textit{cluster} pada antarmuka pengguna di bagian GA.
	\item \textit{spinnerPopulasi}: atribut ini bertipe \textit{Spinner} yang menangani masukan untuk parameter banyaknya populasi pada antarmuka pengguna di bagian GA.
	\item \textit{spinnerMutasi}: atribut ini bertipe \textit{Spinner} yang menangani masukan untuk parameter probabilitas mutasi pada antarmuka pengguna di bagian GA.
	\item \textit{spinnerMaxIterasi}: atribut ini bertipe \textit{Spinner} yang menangani masukan untuk parameter maksimum iterasi pada antarmuka pengguna di bagian GA.
	\item \textit{spinnerElitism}: atribut ini bertipe \textit{Spinner} yang menangani masukan untuk parameter individu elitisme pada antarmuka pengguna di bagian GA.
	\item \textit{spinnerConvergeGen}: atribut ini bertipe \textit{Spinner} yang menangani masukan untuk parameter banyaknya generasi konvergen pada antarmuka pengguna di bagian GA.
	\item \textit{choiceBoxWeighting}: atribut ini bertipe \textit{ChoiceBox} yang menangani masukan untuk parameter metode pembobotan pada antarmuka pengguna di bagian GA.
	\item \textit{spinnerConvergeLimit}: atribut ini bertipe \textit{Spinner} yang menangani masukan untuk parameter banyaknya generasi konvergen pada antarmuka pengguna di bagian GA.
	\item \textit{progressBar}: atribut ini bertipe \textit{progressBar} yang menampilkan perkembangan dari proses pengelompokan menggunakan algoritma GA.
	\item \textit{buttonMulai}: atribut ini bertipe \textit{Button} yang merupakan tombol untuk memulai proses pengelompokan menggunakan algoritma GA.
	\item \textit{textFieldHasil}: atribut ini bertipe \textit{TextField} dan berfungsi menampilkan direktori hasil pada antarmuka pengguna.
	\item \textit{buttonHasil}: atribut ini bertipe \textit{Button} yang merupakan tombol untuk memilih direktori hasil pada antarmuka pengguna.
	\item \textit{tabGA}: atribut ini bertipe \textit{Tab} yang merupakan \textit{tab} untuk memilih pengelompokan menggunakan algoritma GA.
	\item \textit{tabKMeans}: atribut ini bertipe \textit{Tab} yang merupakan \textit{tab} untuk memilih pengelompokan menggunakan algoritma \textit{K-means}.
	\item \textit{labelProgress}: atribut ini bertipe \textit{Label} yang berfungsi untuk menampilkan status dari proses pengelompokan menggunakan algoritma GA.
	\item \textit{KMspinnerCluster}: atribut ini bertipe \textit{Spinner} yang menangani masukan untuk parameter banyaknya \textit{cluster} pada antarmuka pengguna di bagian \textit{K-means}.
	\item \textit{KMchoiceBoxWeighting}: atribut ini bertipe \textit{ChoiceBox} yang menangani masukan untuk parameter metode pembobotan pada antarmuka pengguna di bagian \textit{K-means}.
	\item \textit{KMspinnerMaxIterasi}: atribut ini bertipe \textit{Spinner} yang menangani masukan untuk parameter maksimum iterasi pada antarmuka pengguna di bagian \textit{K-means}.
	\item \textit{KMprogressBar}: atribut ini bertipe \textit{progressBar} yang menampilkan perkembangan dari proses pengelompokan menggunakan algoritma \textit{K-means}.
	\item \textit{KMbuttonMulai}: atribut ini bertipe \textit{Button} yang merupakan tombol untuk memulai proses pengelompokan menggunakan algoritma \textit{K-means}.
	\item \textit{labelProgress}: atribut ini bertipe \textit{Label} yang berfungsi untuk menampilkan status dari proses pengelompokan menggunakan algoritma \textit{K-means}.
	\item \textit{thread}: atribut ini bertipe \textit{Thread} dan merupakan sebuah \textit{thread} yang menjalankan proses pengelompokan.
	\item \textit{runningTime}: atribut ini bertipe \textit{long} dan berfungsi untuk menyimpan lamanya program berjalan dalam milidetik.
	\item \textit{task}: atribut ini bertipe \textit{Task} dan berfungsi untuk menjalankan tugas pengelompokan.
	\item \textit{curIt}: atirbut ini bertipe \textit{int} dan berfungsi untuk menyimpan banyaknya iterasi saat ini.
\end{itemize}

\textit{Method} yang terdapat dalam kelas ini adalah:

\begin{itemize}
	\item \textit{initialize}: \textit{method} ini berfungsi untuk menginisialisasi seluruh atribut dan nilai awalnya pada antarmuka pengguna.
	\item \textit{attachWarning}: \textit{method} ini berfungsi untuk menambahkan \textit{listener} sehingga dapat menampilkan pesan apabila aturan validasi dari suatu masukan dilanggar.
	\item \textit{warningPopulation}: \textit{method} ini berfungsi untuk melakukan pengecekkan terhadap aturan validasi parameter banyaknya populasi dan individu elitisme lalu menampilkan pesan apabila aturan validasi dilanggar.
	\item \textit{warningIteration}: \textit{method} ini berfungsi untuk melakukan pengecekkan terhadap aturan validasi parameter maksimal iterasi dan banyaknya generasi konvergen lalu menampilkan pesan apabila aturan validasi dilanggar.
	\item \textit{chooseDocument}: \textit{method} ini berfungsi untuk memilih direktori dokumen.
	\item \textit{chooseResult}: \textit{method} ini berfungsi untuk memilih direktori hasil.
	\item \textit{reset}: \textit{method} ini berfungsi untuk mengembalikan kondisi dari seluruh objek dalam halaman GA agar bisa kembali digunakan untuk proses pengelompokan selanjutnya.
	\item \textit{KMReset}: \textit{method} ini berfungsi untuk mengembalikan kondisi dari seluruh objek dalam halaman \textit{K-means} agar bisa kembali digunakan untuk proses pengelompokan selanjutnya.
	\item \textit{start}: \textit{method} ini berfungsi untuk memulai proses pengelompokan menggunakan GA.
	\item \textit{KMStart}: \textit{method} ini berfungsi untuk memulai proses pengelompokan menggunakan \textit{K-means}.
	\item \textit{timeFormatter}: \textit{method} ini berfungsi untuk membentuk \textit{string} "\textit{hh} jam \textit{mm} menit \textit{ss} detik" berdasarkan input waktu dalam milidetik.
	\item \textit{writeToFile}: \textit{method} ini berfungsi untuk menulis hasil pengelompokan ke dalam suatu \textit{file} dengan ekstensi CSV.
\end{itemize}

\section{Perancangan Antarmuka Pengguna}
\label{sec:guidesign}
Antarmuka yang dirancang untuk perangkat lunak ini hanya terdiri dari 2 halaman. Rancangan antarmuka ini dibuat sedemikian rupa sehingga memudahkan penggunannya dalam melakukan pengujian terhadap perangkat lunak. Pada penelitian ini, perancangan antarmuka dibuat menggunakan perangkat lunak \textit{balsamiq}\footnote{https://balsamiq.com/}. Setiap objek dan \textit{field} diberi label unik agar dapat disesuaikan dengan tabel keterangan. Berikut dibahas rancangan antarmuka pengguna dari perangkat ini.

\subsection{Halaman Algoritma Genetika}
Gambar \ref{fig:UI-GA} menunjukkan halaman yang dapat digunakan oleh pengguna untuk mengelompokan dokumen menggunakan algoritma genetika. Pada halaman ini terdapat beberapa \textit{field} yang dapat digunakan pengguna untuk mengatur nilai dari masing-masing parameter.

\begin{figure}[H]
	\begin{center}
		\includegraphics[width=0.7\textwidth]{UI/GA-Mockup}
		\caption{Rancangan antarmuka halaman algoritma genetika}
		\label{fig:UI-GA}
	\end{center}
\end{figure}

Penjelasan setiap \textit{field} dalam halaman ini dijelaskan dalam Tabel \ref{tbl:field-GA}.

\begin{table}[H]
	\renewcommand{\arraystretch}{2}
	\caption{Rincian \textit{field} pada halaman algoritma genetika}
	\begin{tabularx}{\textwidth}{l X l X l X} \hline
		\textbf{Kode} & \textbf{Nama} & \textbf{Jenis} & \textbf{\textit{Defaut value}} & \textbf{Wajib} & \textbf{Keterangan} \\ \hline
		A01 & Direktori Dokumen & \textit{file chooser} & - & ya & Folder yang dipilih harus berisi file teks yang akan dikelompokan menggunakan perangkat lunak \\ \hline
		A02 & Direktori Hasil & \textit{file chooser} & - & ya & Folder yang dipilih merupakan folder tempat menyimpan file CSV hasil pengelompokan \\ \hline
		A03 & Banyaknya Cluster & \textit{spinner} & 5 & ya & Nilai minimum 1 \\ \hline
		A04 & Banyaknya Populasi & \textit{spinner} & 1 & ya & Nilai minimum 1 dan harus lebih besar dari Individu Elitisme (A08) \\ \hline
		A05 & Metode Pembobotan & \textit{dropdown} & TF-IDF & ya & Hanya bisa bernilai "TF-IDF" dan "Frekuensi" \\ \hline
		A06 & Probabilitas Mutasi & \textit{spinner} & 0.05 & ya & Nilai minimum 0.01, maksimum 1.00 \\ \hline
		A07 & Maksimum Iterasi & \textit{spinner} & 100 & ya & Nilai minimum 1 dan harus lebih besar dari Banyaknya Generasi Konvergen (A09)\\ \hline
		A08 & Individu Elitisme & \textit{spinner} & 1 & ya & Nilai minimum 0 dan harus lebih kecil dari Banyaknya Populasi (A04)\\ \hline
		A09 & Banyaknya Generasi Konvergen & \textit{spinner} & 3 & ya & Nilai minimum 2 dan harus lebih kecil dari Maksimum Iterasi (A07) \\ \hline
		A10 & Batas Konvergen & \textit{spinner} & 0.00001 & ya & Hanya bisa bernilai $10^{-3}$, $10^{-4}$, $10^{-5}$, $10^{-6}$, dan $10^{-7}$ \\ \hline
	\end{tabularx}
	\label{tbl:field-GA}
\end{table}

Objek dengan kode A11 merupakan sebuah \textit{progress bar} yang menampilkan perkembangan dari jalannya program untuk setiap iterasi. Apabila proses dalam suatu iterasi sudah selesai, maka mengubah nilai dari label A12 dan membuat \textit{progress bar} kembali kosong. Label dengan kode A12 berfungsi untuk menampilkan status dari program yang sedang berjalan. Label ini tidak memiliki nilai apabila program belum dijalankan dan menampilkan status "Inisialisasi..." apabila program sedang melakukan pengindeksan dokumen dan inisialisasi populasi. Setelah iterasi dimulai maka label A12 menampilkan generasi saat ini. Sebagai contoh apabila label A12 menampilkan status "Generasi 1" artinya saat ini program sedang melakukan proses-proses pada generasi 1. Begitu juga untuk "Generasi 2" dan seterusnya. Tombol dengan kode A13 berfungsi untuk memulai proses pengelompokan menggunakan algoritma genetika berdasarkan parameter yang telah dimasukkan.

\subsection{Halaman \textit{K-Means}}
\label{sub:ui-kmeans}
\begin{figure}[H]
	\begin{center}
		\includegraphics[width=0.7\textwidth]{UI/KMeans-mockup}
		\caption{Rancangan antarmuka halaman algoritma genetika}
		\label{fig:UI-KMeans}
	\end{center}
\end{figure}

Gambar \ref{fig:UI-KMeans} merupakan tampilan yang digunakan oleh pengguna untuk melakukan pengelompokan dokumen menggunakan algoritma \textit{K-means}. Berbeda dengan halaman algoritma genetika, pada halaman ini hanya terdapat tiga buah parameter yang dapat diubah-ubah oleh pengguna dalam melakukan pengelompokan.

\begin{table}[H]
	\renewcommand{\arraystretch}{2}
	\caption{Rincian \textit{field} pada halaman algoritma \textit{K-means}}
	\begin{tabularx}{\textwidth}{l X l X l X} \hline
		\textbf{Kode} & \textbf{Nama} & \textbf{Jenis} & \textbf{\textit{Defaut value}} & \textbf{Wajib} & \textbf{Keterangan} \\ \hline
		B01 & Direktori Dokumen & \textit{file chooser} & - & ya & Folder yang dipilih harus berisi file teks yang akan dikelompokan menggunakan perangkat lunak \\ \hline
		B02 & Direktori Hasil & \textit{file chooser} & - & ya & Folder yang dipilih merupakan folder tempat menyimpan file CSV hasil pengelompokan \\ \hline
		B03 & Banyaknya Cluster & \textit{spinner} & 5 & ya & Nilai minimum 1 \\ \hline
		B04 & Metode Pembobotan & \textit{dropdown} & TF-IDF & ya & Hanya bisa bernilai "TF-IDF" dan "Frekuensi" \\ \hline
		B05 & Maksimum Iterasi & \textit{spinner} & 100 & ya & Nilai minimum 1\\ \hline
	\end{tabularx}
	\label{tbl:field-KMeans}
\end{table}

Dalam Tabel \ref{tbl:field-KMeans}, objek dengan kode B06 merupakan sebuah \textit{progress bar} yang menampilkan perkembangan dari jalannya program yang menyatakan banyaknya iterasi yang sudah selesai dijalankan dibandingkan dengan maksimum iterasi. Label B07 berisi keterangan dari proses yang sedang berlangsung. Salah satu contoh isi dari label B07 adalah "Iterasi 2/100" yang berarti saat ini sedang dilakukan pemrosesan pada iterasi kedua dari 100 iterasi. Tombol dengan kode B08 berfungsi untuk memulai proses pengelompokan menggunakan algoritma \textit{K-means} berdasarkan parameter yang telah dimasukkan.
