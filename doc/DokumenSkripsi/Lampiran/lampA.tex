%versi 3 (18-12-2016)
\chapter{Kode Program}
\label{lamp:A}

%terdapat 2 cara untuk memasukkan kode program
% 1. menggunakan perintah \lstinputlisting (kode program ditempatkan di folder yang sama dengan file ini)
% 2. menggunakan environment lstlisting (kode program dituliskan di dalam file ini)
% Perhatikan contoh yang diberikan!!
%
% untuk keduanya, ada parameter yang harus diisi:
% - language: bahasa dari kode program (pilihan: Java, C, C++, PHP, Matlab, C#, HTML, R, Python, SQL, dll)
% - caption: nama file dari kode program yang akan ditampilkan di dokumen akhir
%
% Perhatian: Abaikan warning tentang textasteriskcentered!!
%


\lstinputlisting[language=Xml, caption=FXMLDocument.fxml]{./Code/FXMLDocument.fxml}
\lstinputlisting[language=Java, caption=FXMLDocumentController.java]{./Code/FXMLDocumentController.java}
\lstinputlisting[language=Java, caption=Params.java]{./Code/Params.java}
\lstinputlisting[language=Java, caption=Chromosome.java]{./Code/GA/Chromosome.java}
\lstinputlisting[language=Java, caption=GAClusterer.java]{./Code/GA/GAClusterer.java}
\lstinputlisting[language=Java, caption=Gene.java]{./Code/GA/Gene.java}
\lstinputlisting[language=Java, caption=CosineSimilarityCalculator.java]{./Code/IR/CosineSimilarityCalculator.java}
\lstinputlisting[language=Java, caption=Document.java]{./Code/IR/Document.java}
\lstinputlisting[language=Java, caption=FrequencyWeighting.java]{./Code/IR/FrequencyWeighting.java}
\lstinputlisting[language=Java, caption=Lexicon.java]{./Code/IR/Lexicon.java}
\lstinputlisting[language=Java, caption=SimilarityCalculator.java]{./Code/IR/SimilarityCalculator.java}
\lstinputlisting[language=Java, caption=TermWeighting.java]{./Code/IR/TermWeighting.java}
\lstinputlisting[language=Java, caption=TFIDFWeighting.java]{./Code/IR/TFIDFWeighting.java}
\lstinputlisting[language=Java, caption=Vector.java]{./Code/IR/Vector.java}
\lstinputlisting[language=Java, caption=KMeans.java]{./Code/KMeans/KMeans.java}

