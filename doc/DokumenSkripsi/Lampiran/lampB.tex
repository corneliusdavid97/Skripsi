%versi 2 (8-10-2016)
\chapter{Hasil Eksperimen}
\label{lamp:B}

Hasil eksperimen algoritma genetika:
\begin{enumerate}
	\item Kasus uji 1 (parameter ideal)\\
	Parameter:
	\begin{itemize}
		\item Banyaknya \textit{Cluster}: 5
		\item Banyaknya Populasi: 100
		\item Metode Pembobotan: TF-IDF
		\item Probabilitas Mutasi: 0.05
		\item Maksimum Iterasi it: 100
		\item Individu Elitisme: 1
		\item Banyaknya Generasi Konvergen: 3
		\item Batas Konvergen: 0.00001
	\end{itemize}
	Hasil eksperimen berdasarkan parameter diatas dijelaskan dalam Tabel \ref{tbl:res1}.
	\begin{table}[H]
		\centering
		\begin{tabular}{|l|l|l|l|} \hline
			Waktu & Intercluster & Iterasi & Purity \\ \hline
			1 jam 2 menit 9 detik  & 781.4455224 & 5 & 0.802696629 \\ \hline
			1 jam 5 menit 3 detik  & 533.7897158 & 5 & 0.820224719 \\ \hline
			1 jam 1 menit 31 detik & 781.7710303 & 4 & 0.794157303 \\ \hline
			1 jam 9 menit 21 detik & 781.5705197 & 5 & 0.78741573  \\ \hline
			1 jam 9 menit 44 detik & 779.5194513 & 4 & 0.78741573 \\ \hline
		\end{tabular}
		\caption{Hasil eksperimen kasus uji 1 (parameter ideal)}
		\label{tbl:res1}
	\end{table}
	
	\item Kasus uji 2 (Populasi=50)\\
	Parameter:
	\begin{itemize}
		\item Banyaknya \textit{Cluster}: 5
		\item Banyaknya Populasi: 50
		\item Metode Pembobotan: TF-IDF
		\item Probabilitas Mutasi: 0.05
		\item Maksimum Iterasi it: 100
		\item Individu Elitisme: 1
		\item Banyaknya Generasi Konvergen: 3
		\item Batas Konvergen: 0.00001
	\end{itemize}
	Hasil eksperimen berdasarkan parameter diatas dijelaskan dalam Tabel \ref{tbl:res2}.
	\begin{table}[H]
		\centering
		\begin{tabular}{|l|l|l|l|} \hline
			Waktu & Intercluster & Iterasi & Purity \\ \hline
			34 menit 49 detik & 537.8835842 & 4 & 0.770786517 \\ \hline
			50 menit 13 detik & 922.3475657 & 6 & 0.708764045 \\ \hline
			54 menit 6 detik  & 782.7303944 & 6 & 0.806292135 \\ \hline
			37 menit 26 detik & 779.3514206 & 4 & 0.785617978 \\ \hline
			35 menit 36 detik & 780.2768031 & 4 & 0.826516854 \\ \hline
		\end{tabular}
		\caption{Hasil eksperimen kasus uji 2 (Populasi=50)}
		\label{tbl:res2}
	\end{table}
	
	\item Kasus uji 3 (Populasi=150)\\
	Parameter:
	\begin{itemize}
		\item Banyaknya \textit{Cluster}: 5
		\item Banyaknya Populasi: 150
		\item Metode Pembobotan: TF-IDF
		\item Probabilitas Mutasi: 0.05
		\item Maksimum Iterasi it: 100
		\item Individu Elitisme: 1
		\item Banyaknya Generasi Konvergen: 3
		\item Batas Konvergen: 0.00001
	\end{itemize}
	Hasil eksperimen berdasarkan parameter diatas dijelaskan dalam Tabel \ref{tbl:res3}.
	\begin{table}[H]
		\centering
		\begin{tabular}{|l|l|l|l|} \hline
			Waktu & Intercluster & Iterasi & Purity \\ \hline
			1 jam 40 menit 56 detik & 781.0071837 & 4 & 0.758651685 \\ \hline
			1 jam 38 menit 5 detik  & 545.8893228 & 4 & 0.96        \\ \hline
			1 jam 22 menit 4 detik  & 920.7411105 & 4 & 0.623820225 \\ \hline
			2 jam 10 menit 59 detik & 542.7638178 & 5 & 0.965393258 \\ \hline
			2 jam 25 menit 13 detik & 1520.496586 & 6 & 0.425617978 \\ \hline
		\end{tabular}
		\caption{Hasil eksperimen kasus uji 3 (Populasi=150)}
		\label{tbl:res3}
	\end{table}
	
	\item Kasus uji 4 (Bobot frekuensi)\\
	Parameter:
	\begin{itemize}
		\item Banyaknya \textit{Cluster}: 5
		\item Banyaknya Populasi: 100
		\item Metode Pembobotan: Frekuensi
		\item Probabilitas Mutasi: 0.05
		\item Maksimum Iterasi it: 100
		\item Individu Elitisme: 1
		\item Banyaknya Generasi Konvergen: 3
		\item Batas Konvergen: 0.00001
	\end{itemize}
	Hasil eksperimen berdasarkan parameter diatas dijelaskan dalam Tabel \ref{tbl:res4}.
	\begin{table}[H]
		\centering
		\begin{tabular}{|l|l|l|l|} \hline
			Waktu & Intercluster & Iterasi & Purity \\ \hline
			2 jam 44 menit 10 detik & 1750.439783 & 5 & 0.581123596 \\ \hline
			2 jam 42 menit 52 detik & 1750.902724 & 8 & 0.57258427  \\ \hline
			2 jam 13 menit 36 detik & 1750.920796 & 6 & 0.543370787 \\ \hline
			2 jam 34 menit 57 detik & 1751.049984 & 7 & 0.583820225 \\ \hline
			3 jam 24 menit 23 detik & 1751.061378 & 9 & 0.575730337 \\ \hline
		\end{tabular}
		\caption{Hasil eksperimen kasus uji 4 (Bobot frekuensi)}
		\label{tbl:res4}
	\end{table}
	
	\item Kasus uji 5 (Probabilitas mutasi=0)\\
	Parameter:
	\begin{itemize}
		\item Banyaknya \textit{Cluster}: 5
		\item Banyaknya Populasi: 100
		\item Metode Pembobotan: TF-IDF
		\item Probabilitas Mutasi: 0
		\item Maksimum Iterasi it: 100
		\item Individu Elitisme: 1
		\item Banyaknya Generasi Konvergen: 3
		\item Batas Konvergen: 0.00001
	\end{itemize}
	Hasil eksperimen berdasarkan parameter diatas dijelaskan dalam Tabel \ref{tbl:res5}.
	\begin{table}[H]
		\centering
		\begin{tabular}{|l|l|l|l|} \hline
			Waktu & Intercluster & Iterasi & Purity \\ \hline
			2 jam 11 menit 4 detik  & 779.4871027 & 6 & 0.893932584 \\ \hline
			1 jam 14 menit 37 detik & 543.1994982 & 4 & 0.955955056 \\ \hline
			1 jam 54 menit 46 detik & 1610.403312 & 6 & 0.361348315 \\ \hline
			1 jam 1 menit 17 detik  & 920.2416313 & 4 & 0.595955056 \\ \hline
			1 jam 53 menit 29 detik & 1675.987642 & 6 & 0.354606742 \\ \hline
		\end{tabular}
		\caption{Hasil eksperimen kasus uji 5 (Probabilitas mutasi=0)}
		\label{tbl:res5}
	\end{table}
	
	\item Kasus uji 6 Probabilitas mutasi=0.25)\\
	Parameter:
	\begin{itemize}
		\item Banyaknya \textit{Cluster}: 5
		\item Banyaknya Populasi: 100
		\item Metode Pembobotan: TF-IDF
		\item Probabilitas Mutasi: 0.25
		\item Maksimum Iterasi it: 100
		\item Individu Elitisme: 1
		\item Banyaknya Generasi Konvergen: 3
		\item Batas Konvergen: 0.00001
	\end{itemize}
	Hasil eksperimen berdasarkan parameter diatas dijelaskan dalam Tabel \ref{tbl:res6}.
	\begin{table}[H]
		\centering
		\begin{tabular}{|l|l|l|l|} \hline
			Waktu & Intercluster & Iterasi & Purity \\ \hline
			1 jam 16 menit 39 detik & 1206.317545 & 5 & 0.45752809  \\ \hline
			1 jam 32 menit 52 detik & 1393.180775 & 6 & 0.414382022 \\ \hline
			1 jam 48 menit 20 detik & 921.4314001 & 5 & 0.567191011 \\ \hline
			1 jam 23 menit 52 detik & 1204.926873 & 4 & 0.344719101 \\ \hline
			1 jam 28 menit 9 detik  & 1726.000931 & 5 & 0.401797753 \\ \hline
		\end{tabular}
		\caption{Hasil eksperimen kasus uji 6 (Probabilitas mutasi=0.25)}
		\label{tbl:res6}
	\end{table}
	
	\item Kasus uji 7 (Individu eltisme=0)\\
	Parameter:
	\begin{itemize}
		\item Banyaknya \textit{Cluster}: 5
		\item Banyaknya Populasi: 100
		\item Metode Pembobotan: TF-IDF
		\item Probabilitas Mutasi: 0.05
		\item Maksimum Iterasi it: 100
		\item Individu Elitisme: 0
		\item Banyaknya Generasi Konvergen: 3
		\item Batas Konvergen: 0.00001
	\end{itemize}
	Hasil eksperimen berdasarkan parameter diatas dijelaskan dalam Tabel \ref{tbl:res7}.
	\begin{table}[H]
		\centering
		\begin{tabular}{|l|l|l|l|} \hline
			Waktu & Intercluster & Iterasi & Purity \\ \hline
			5 jam 55 menit 40 detik & 775.2651111 & 15 & 0.789213483 \\ \hline
			6 jam 3 menit 59 detik  & 773.4957837 & 15 & 0.800898876 \\ \hline
			6 jam 18 menit 24 detik & 775.5719472 & 15 & 0.795505618 \\ \hline
			5 jam 42 menit 27 detik & 540.8704844 & 15 & 0.977078652 \\ \hline
			5 jam 48 menit 31 detik & 533.7854398 & 15 & 0.768539326 \\ \hline
		\end{tabular}
		\caption{Hasil eksperimen kasus uji 7 (Individu elitisme=0)}
		\label{tbl:res7}
	\end{table}

	\item Kasus uji 8 (Individu elitisme=5)\\
	Parameter:
	\begin{itemize}
		\item Banyaknya \textit{Cluster}: 5
		\item Banyaknya Populasi: 100
		\item Metode Pembobotan: TF-IDF
		\item Probabilitas Mutasi: 0.05
		\item Maksimum Iterasi it: 100
		\item Individu Elitisme: 5
		\item Banyaknya Generasi Konvergen: 3
		\item Batas Konvergen: 0.00001
	\end{itemize}
	Hasil eksperimen berdasarkan parameter diatas dijelaskan dalam Tabel \ref{tbl:res8}.
	\begin{table}[H]
		\centering
		\begin{tabular}{|l|l|l|l|} \hline
			Waktu & Intercluster & Iterasi & Purity \\ \hline
			2 jam 5 menit 57 detik  & 1206.637007 & 6 & 0.379775281 \\ \hline
			2 jam 16 menit 44 detik & 1725.804155 & 6 & 0.315955056 \\ \hline
			2 jam 32 menit 16 detik & 1676.015285 & 7 & 0.360898876 \\ \hline
			2 jam 3 menit 35 detik  & 545.358012  & 6 & 0.966292135 \\ \hline
			1 jam 26 menit 0 detik  & 922.8631224 & 5 & 0.61258427 \\ \hline
		\end{tabular}
		\caption{Hasil eksperimen kasus uji 8 (Individu elitisme=5)}
		\label{tbl:res8}
	\end{table}
\end{enumerate}

Hasil eksperimen algoritma K-means:

Parameter:
\begin{itemize}
	\item Banyaknya \textit{Cluster}: 5
	\item Metode Pembobotan: TF-IDF
	\item Maksimum Iterasi it: 100
\end{itemize}

Hasil eksperimen berdasarkan parameter diatas dijelaskan dalam Tabel \ref{tbl:resKM}.

\begin{table}[H]
	\centering
	\begin{tabular}{|l|l|l|l|} \hline
		Waktu & Intercluster & Iterasi & Purity \\ \hline
		1 menit 30 detik & 779.1467826 & 12 & 0.734382022 \\ \hline
		1 menit 2 detik  & 512.6894755 & 7  & 0.456179775 \\ \hline
		45 detik         & 903.4065026 & 7  & 0.252080274 \\ \hline
		1 menit 5 detik  & 1208.220743 & 10 & 0.348314607 \\ \hline
		2 menit 19 detik & 1383.188573 & 22 & 0.276853933 \\ \hline
		1 menit 32 detik & 776.9612952 & 12 & 0.625168539 \\ \hline
		2 menit 12 detik & 543.139087  & 14 & 0.746966292 \\ \hline
		1 menit 31 detik & 785.1360162 & 10 & 0.794606742 \\ \hline
		1 menit 47 detik & 523.7897906 & 13 & 0.636404494 \\ \hline
		58 detik         & 900.2035951 & 8  & 0.248153619 \\ \hline
	\end{tabular}
	\caption{Hasil eksperimen algoritma K-means}
	\label{tbl:resKM}
\end{table}
