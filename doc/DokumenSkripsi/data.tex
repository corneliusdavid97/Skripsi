%_____________________________________________________________________________
%=============================================================================
% data.tex v10 (22-01-2017) dibuat oleh Lionov - T. Informatika FTIS UNPAR
%
% Perubahan pada versi 10 (22-01-2017)
%	- Penambahan overfullrule untuk memeriksa warning
%  	- perubahan mode buku menjadi 4: bimbingan, sidang(1), sidang akhir dan 
%     buku final
%	- perbaikan perintah pada beberapa bagian
%  	- perubahan pengisian tulisan "daftar isi" yang error
%  	- penghilangan lipsum dari file ini
%_____________________________________________________________________________
%=============================================================================

%=============================================================================
% 								PETUNJUK
%=============================================================================
% Ini adalah file data (data.tex)
% Masukkan ke dalam file ini, data-data yang diperlukan oleh template ini
% Cara memasukkan data dijelaskan di setiap bagian
% Data yang WAJIB dan HARUS diisi dengan baik dan benar adalah SELURUHNYA !!
% Hilangkan tanda << dan >> jika anda menemukannya
%=============================================================================

%_____________________________________________________________________________
%=============================================================================
% 								BAGIAN 0
%=============================================================================
% Entri untuk memperbaiki posisi "DAFTAR ISI" jika tidak berada di bagian 
% tengah halaman. Sayangnya setiap sistem menghasilkan posisi yang berbeda.
% Isilah dengan 0 atau 1 (e.g. \daftarIsiError{1}). 
% Pemilihan 0 atau 1 silahkan disesuaikan dengan hasil PDF yang dihasilkan.
%=============================================================================
\daftarIsiError{0}   
%\daftarIsiError{1}   
%=============================================================================

%_____________________________________________________________________________
%=============================================================================
% 								BAGIAN I
%=============================================================================
% Tambahkan package2 lain yang anda butuhkan di sini
%=============================================================================
\usepackage{booktabs} 
\usepackage{longtable}
\usepackage{amssymb}
\usepackage{todo}
\usepackage{verbatim} 		%multiline comment
\usepackage{pgfplots}
\usepackage{algcompatible}
\usepackage{algorithm}
\usepackage{amsmath}
\usepackage{tabularx}
\usepackage{xcolor}
\usepackage{subcaption}
\usepackage{multirow}

%\overfullrule=3mm % memperlihatkan overfull 
%=============================================================================

%_____________________________________________________________________________
%=============================================================================
% 								BAGIAN II
%=============================================================================
% Mode dokumen: menetukan halaman depan dari dokumen, apakah harus mengandung 
% prakata/pernyataan/abstrak dll (termasuk daftar gambar/tabel/isi) ?
% - final 		: hanya untuk buku skripsi, dicetak lengkap: judul ina/eng, 
%   			  pengesahan, pernyataan, abstrak ina/eng, untuk, kata 
%				  pengantar, daftar isi (daftar tabel dan gambar tetap 
%				  opsional dan dapat diatur), seluruh bab dan lampiran.
%				  Otomatis tidak ada nomor baris dan singlespacing
% - sidangakhir	: buku sidang akhir = buku final - (pengesahan + pernyataan +
%   			  untuk + kata pengantar)
%				  Otomatis ada nomor baris dan onehalfspacing 
% - sidang 		: untuk sidang 1, buku sidang = buku sidang akhir - (judul 
%				  eng + abstrak ina/eng)
%				  Otomatis ada nomor baris dan onehalfspacing
% - bimbingan	: untuk keperluan bimbingan, hanya terdapat bab dan lampiran
%   			  saja, bab dan lampiran yang hendak dicetak dapat ditentukan 
%				  sendiri (nomor baris dan spacing dapat diatur sendiri)
% Mode default adalah 'template' yang menghasilkan isian berwarna merah, 
% aktifkan salah satu mode di bawah ini :
%=============================================================================
%\mode{bimbingan} 		% untuk keperluan bimbingan
%\mode{sidang} 			% untuk sidang 1
%\mode{sidangakhir} 	% untuk sidang 2 / sidang pada Skripsi 2(IF)
\mode{final} 			% untuk mencetak buku skripsi 
%=============================================================================

%_____________________________________________________________________________
%=============================================================================
% 								BAGIAN III
%=============================================================================
% Line numbering: penomoran setiap baris, nomor baris otomatis di-reset ke 1
% setiap berganti halaman.
% Sudah dikonfigurasi otomatis untuk mode final (tidak ada), mode sidang (ada)
% dan mode sidangakhir (ada).
% Untuk mode bimbingan, defaultnya ada (\linenumber{yes}), jika ingin 
% dihilangkan, isi dengan "no" (i.e.: \linenumber{no})
% Catatan:
% - jika nomor baris tidak kembali ke 1 di halaman berikutnya, compile kembali
%   dokumen latex anda
% - bagian ini hanya bisa diatur di mode bimbingan
%=============================================================================
%\linenumber{no} 
\linenumber{yes}
%=============================================================================

%_____________________________________________________________________________
%=============================================================================
% 								BAGIAN IV
%=============================================================================
% Linespacing: jarak antara baris 
% - single	: otomatis jika ingin mencetak buku skripsi, opsi yang 
%			     disediakan untuk bimbingan, jika pembimbing tidak keberatan 
%			     (untuk menghemat kertas)
% - onehalf	: otomatis jika ingin mencetak dokumen untuk sidang
% - double 	: jarak yang lebih lebar lagi, jika pembimbing berniat memberi 
%             catatan yg banyak di antara baris (dianjurkan untuk bimbingan)
% Catatan: bagian ini hanya bisa diatur di mode bimbingan
%=============================================================================
\linespacing{single}
%\linespacing{onehalf}
%\linespacing{double}
%=============================================================================

%_____________________________________________________________________________
%=============================================================================
% 								BAGIAN V
%=============================================================================
% Tidak semua skripsi memuat gambar dan/atau tabel. Untuk skripsi yang tidak 
% memiliki gambar dan/atau tabel, maka tidak diperlukan Daftar Gambar dan/atau 
% Daftar Tabel. Sayangnya hal tsb sulit dilakukan secara manual karena 
% membutuhkan kedisiplinan pengguna template.  
% Jika tidak ingin menampilkan Daftar Gambar dan/atau Daftar Tabel, karena 
% tidak ada gambar atau tabel atau karena dokumen dicetak hanya untuk 
% bimbingan, isi dengan "no" (e.g. \gambar{no})
%=============================================================================
\gambar{yes}
%\gambar{no}
\tabel{yes}
%\tabel{no}  
%=============================================================================

%_____________________________________________________________________________
%=============================================================================
% 								BAGIAN VI
%=============================================================================
% Pada mode final, sidang da sidangkahir, seluruh bab yang ada di folder "Bab"
% dengan nama file bab1.tex, bab2.tex s.d. bab9.tex akan dicetak terurut, 
% apapun isi dari perintah \bab.
% Pada mode bimbingan, jika ingin:
% - mencetak seluruh bab, isi dengan 'all' (i.e. \bab{all})
% - mencetak beberapa bab saja, isi dengan angka, pisahkan dengan ',' 
%   dan bab akan dicetak terurut sesuai urutan penulisan (e.g. \bab{1,3,2}). 
% Catatan: Jika ingin menambahkan bab ke-3 s.d. ke-9, tambahkan file 
% bab3.tex, bab4.tex, dst di folder "Bab". Untuk bab ke-10 dan 
% seterusnya, harus dilakukan secara manual dengan mengubah file skripsi.tex
% Catatan: bagian ini hanya bisa diatur di mode bimbingan
%=============================================================================
\bab{all}
%=============================================================================

%_____________________________________________________________________________
%=============================================================================
% 								BAGIAN VII
%=============================================================================
% Pada mode final, sidang dan sidangkhir, seluruh lampiran yang ada di folder 
% "Lampiran" dengan nama file lampA.tex, lampB.tex s.d. lampJ.tex akan dicetak 
% terurut, apapun isi dari perintah \lampiran.
% Pada mode bimbingan, jika ingin:
% - mencetak seluruh lampiran, isi dengan 'all' (i.e. \lampiran{all})
% - mencetak beberapa lampiran saja, isi dengan huruf, pisahkan dengan ',' 
%   dan lampiran akan dicetak terurut sesuai urutan (e.g. \lampiran{A,E,C}). 
% - tidak mencetak lampiran apapun, isi dengan "none" (i.e. \lampiran{none})
% Catatan: Jika ingin menambahkan lampiran ke-C s.d. ke-I, tambahkan file 
% lampC.tex, lampD.tex, dst di folder Lampiran. Untuk lampiran ke-J dan 
% seterusnya, harus dilakukan secara manual dengan mengubah file skripsi.tex
% Catatan: bagian ini hanya bisa diatur di mode bimbingan
%=============================================================================
\lampiran{all}
%=============================================================================

%_____________________________________________________________________________
%=============================================================================
% 								BAGIAN VIII
%=============================================================================
% Data diri dan skripsi/tugas akhir
% - namanpm		: Nama dan NPM anda, penggunaan huruf besar untuk nama harus 
%				  benar dan gunakan 10 digit npm UNPAR, PASTIKAN BAHWA 
%				  BENAR !!! (e.g. \namanpm{Jane Doe}{1992710001}
% - judul 		: Dalam bahasa Indonesia, perhatikan penggunaan huruf besar, 
%				  judul tidak menggunakan huruf besar seluruhnya !!! 
% - tanggal 	: isi dengan {tangga}{bulan}{tahun} dalam angka numerik, 
%				  jangan menuliskan kata (e.g. AGUSTUS) dalam isian bulan.
%			  	  Tanggal ini adalah tanggal dimana anda akan melaksanakan 
%				  sidang ujian akhir skripsi/tugas akhir
% - pembimbing	: pembimbing anda, lihat daftar dosen di file dosen.tex
%				  jika pembimbing hanya 1, kosongkan parameter kedua 
%				  (e.g. \pembimbing{\JND}{} ), \JND adalah kode dosen
% - penguji 	: para penguji anda, lihat daftar dosen di file dosen.tex
%				  (e.g. \penguji{\JHD}{\JCD} )
% !!Lihat singkatan pembimbing dan penguji anda di file dosen.tex!!
% Petunjuk: hilangkan tanda << & >>, dan isi sesuai dengan data anda
%=============================================================================
\namanpm{Cornelius David Herianto}{2015730034}
\tanggal{22}{5}{2019}
\pembimbing{\KDH}{}    
\penguji{\HUH}{\RDL} 
%=============================================================================

%_____________________________________________________________________________
%=============================================================================
% 								BAGIAN IX
%=============================================================================
% Judul dan title : judul bhs indonesia dan inggris
% - judulINA: judul dalam bahasa indonesia
% - judulENG: title in english
% Petunjuk: 
% - hilangkan tanda << & >>, dan isi sesuai dengan data anda
% - langsung mulai setelah '{' awal, jangan mulai menulis di baris bawahnya
% - gunakan \texorpdfstring{\\}{} untuk pindah ke baris baru
% - judul TIDAK ditulis dengan menggunakan huruf besar seluruhnya !!
%=============================================================================
\judulINA{Pengelompokan Dokumen Berbasis Algoritma Genetika}
\judulENG{GA-Based Document Clustering}
%_____________________________________________________________________________
%=============================================================================
% 								BAGIAN X
%=============================================================================
% Abstrak dan abstract : abstrak bhs indonesia dan inggris
% - abstrakINA: abstrak bahasa indonesia
% - abstrakENG: abstract in english 
% Petunjuk: 
% - hilangkan tanda << & >>, dan isi sesuai dengan data anda
% - langsung mulai setelah '{' awal, jangan mulai menulis di baris bawahnya
%=============================================================================
\abstrakINA{Pengelompokan (\textit{clustering}) merupakan sebuah metode untuk menggabungkan himpunan objek ke dalam kelompok-kelompok sedemikan rupa sehingga objek dalam kelompok (\textit{cluster}) lebih mirip (karena suatu hal) satu sama lain daripada objek di kelompok lain \cite{gan2007data}. \textit{Document clustering} (pengelompokan dokumen) merupakan proses pengelompokan yang dilakukan terhadap suatu koleksi dokumen. Pengelompokan dokumen diterapkan dalam beberapa bidang seperti penambangan web, mesin pencari (\textit{search engine}), dan temu kembali informasi (\textit{information retrieval}) \cite{shah2012document}. Hal yang dilakukan dalam pengelompokan dokumen adalah mengukur kemiripan (\textit{similarity}) antar dokumen dan mengelompokan dokumen yang serupa. Salah satu algoritma pengelompokan yang paling sering digunakan adalah \textit{K-means}. Namun, algoritma \textit{K-means} memiliki kekurangan yaitu dapat terjebak dalam \textit{local optimum}. \textit{Local optimum} adalah suatu solusi yang optimal (baik maksimal maupun minimal) diantara kandidat solusi yang berdekatan dalam masalah optimasi. Dikatakan lokal karena solusi ini hanya optimal apabila dibandingkan dengan kandidat solusi yang berdekatan, tidak optimal secara keseluruhan (\textit{global optimum}).

Algoritma genetika atau biasa disebut \textit{Genetic Algorithm} (GA) adalah suatu algoritma pencarian yang terinspirasi dari proses seleksi alam yang terjadi secara alami dalam proses evolusi. GA merupakan metode penyelesaian masalah yang menggunakan genetika sebagai pemodelannya. Suatu calon solusi dalam GA dimodelkan sebagai suatu individu. Kumpulan individu-individu ini disebut dengan populasi. Setiap individu dalam populasi direpresentasikan dengan kromosom. Kromosom merupakan kumpulan parameter yang membentuk suatu solusi. Parameter-parameter yang menyusun kromosom disebut dengan gen. Setiap kromosom memiliki suatu nilai yang terkait dengan \textit{fitness} dari solusi yang direpresentasikannya. Nilai itu biasanya disebut dengan nilai \textit{fitness}. Dalam penelitian ini, GA akan digunakan sebagai solusi dari masalah \textit{local optimum}. \textit{Local optimum} dapat diatasi oleh GA yang sudah terbukti efektif dalam masalah pencarian dan optimasi. GA dapat digunakan untuk mengelompokkan dokumen dengan beberapa adaptasi terhadap representasi kromosom, fungsi \textit{fitness}, seleksi, persilangan, dan mutasi. 

Algoritma genetika dan algoritma \textit{K-means} diuji menggunakan suatu \textit{dataset} berlabel untuk membandingkan waktu dan hasil pengelompokan dari kedua algoritma tersebut. Berdasarkan hasil eksperimen menggunakan \textit{dataset} dalam penelitian ini, rata-rata nilai \textit{purity} dari hasil pengelompokan menggunakan algoritma genetika adalah sebesar 0.799, lebih baik 56\% dibandingkan dengan menggunakan algoritma K-means. Hal ini membuktikan bahwa algoritma genetika sudah dapat mengelompokan dokumen dengan hasil yang memuaskan. Namun dari segi waktu, algoritma genetika membutuhkan waktu 4365\% lebih lama dibandingkan dengan algoritma \textit{K-means}. Hal ini disebabkan oleh proses komputasi yang dilakukan pada algoritma genetika jauh lebih banyak dan kompleks dibandingkan dengan algoritma \textit{K-means}.}

\abstrakENG{Clustering is a method of creating groups of objects, or clusters, in such a way that objects in one cluster are very similar and objects in different clusters are quite distinct \cite{gan2007data}. Document clustering is an organization of documents into clusters. It has been studied intensively because of its wide applicability in various areas such as web mining, search engines, and information retrieval \cite{shah2012document}. It is measuring similarity between documents and grouping similar documents together. One of the most frequently used algorithm in clustering is K-means. However, K-means can easily stuck in local optimum. Local optimum of an optimization problem is a solution that is optimal (either maximal or minimal) within a neighboring set of candidate solutions. This is in contrast to a global optimum, which is the optimal solution among all possible solutions, not just those in a particular neighborhood of values.

Genetic Algorithm (GA) is a search algorithm inspired by the natural selection process that occurs naturally in the evolutionary process. GA is a problem solving method that used genetics as its model. A solution candidate is modeled as an individual in GA. Set of these individuals are called population. Every individual in a population is represented by a chromosome. Chromosome is a collection of parameters which formed a solution. That parameters is called gene. Every individual in the population is assigned, by means of a fitness function, a measure of its goodness In this study, GA will be used as a solution to local optimum, which have proven to be effective in search and optimization problems. GA can be used to cluster documents by adapting chromosome representation, fitness function, selection, crossover, and mutation. 

Genetic algorithm and K-means algorithm will be tested using a labeled datasets to compare the running time and clustering result. Based on the experimental results using the datasets in this study, the average purity value of clustering results using genetic algorithms is 0.799, 56\% greater than using the K-means algorithm. This proves that the genetic algorithm is able to cluster documents with satisfactory results. But in terms of running time, genetic algorithms take 4365\% more time than the K-means algorithm. This is caused by the computational process carried out on the genetic algorithm is far more complex than the K-means algorithm.}
%=============================================================================

%_____________________________________________________________________________
%=============================================================================
% 								BAGIAN XI
%=============================================================================
% Kata-kata kunci dan keywords : diletakkan di bawah abstrak (ina dan eng)
% - kunciINA: kata-kata kunci dalam bahasa indonesia
% - kunciENG: keywords in english
% Petunjuk: hilangkan tanda << & >>, dan isi sesuai dengan data anda.
%=============================================================================
\kunciINA{Algoritma genetika, Pengelompokan dokumen, Algoritma \textit{K-means}, TF-IDF, \textit{Local optimum}}
\kunciENG{Genetic algorithm, Document clustering, K-means algorithm, TF-IDF, Local optimum}
%=============================================================================

%_____________________________________________________________________________
%=============================================================================
% 								BAGIAN XII
%=============================================================================
% Persembahan : kepada siapa anda mempersembahkan skripsi ini ...
% Petunjuk: hilangkan tanda << & >>, dan isi sesuai dengan data anda.
%=============================================================================
\untuk{Dipersembahkan kepada Tuhan YME, keluarga tercinta, \\ dan diri sendiri}
%=============================================================================

%_____________________________________________________________________________
%=============================================================================
% 								BAGIAN XIII
%=============================================================================
% Kata Pengantar: tempat anda menuliskan kata pengantar dan ucapan terima 
% kasih kepada yang telah membantu anda bla bla bla ....  
% Petunjuk: hilangkan tanda << & >>, dan isi sesuai dengan data anda.
%=============================================================================
\prakata{Puji syukur kepada Tuhan Yang Maha Esa atas berkat yang diberikan kepada penulis sehingga dapat menyelesaikan skripsi dengan judul \textbf{Pengelompokan Dokumen Berbasis Algoritma Genetika} dengan baik dan tepat waktu. Selama menjalani proses perkuliahan dan penyusunan skripsi, penulis telah mendapat banyak bantuan dan dukungan dalam menghadapi hambatan yang ada. Oleh karena itu, penulis inign menyampaikan rasa terima kasih kepada:
	\begin{enumerate}
		\item Keluarga penulis yaitu Papa dan Mama yang selalu mendukung penulis secara moral dan materiil sehingga penulis dapat menyelesaikan proses perkuliahan dan skripsi ini dengan baik. Serta kepada Michelle dan Vincent sebagai adik penulis yang selalu mendukung penulis dalam menyelesaikan skripsi ini.
		\item Bapak Kristopher David Harjono, M.T. sebagai dosen pembimbing yang telah membimbing penulis hingga dapat menyelesaikan skripsi ini.
		\item Bapak Husnul Hakim, M.T. dan Ibu Rosa De Lima, M.Kom. sebagai dosen penguji yang telah membantu dalam menguji dan memperbaiki skripsi ini.
		\item Ibu Mariskha Tri Adithia, P.D.Eng selaku Ketua Program Studi Teknik Informatika Fakultas Teknologi Informasi dan Sains Universitas Katolik Parahyangan.
		\item Arlin Sasqia Puspa Shiffa sebagai sahabat yang selalu mendampingi penulis dalam penyusunan skripsi dari awal hingga akhir bahkan membantu penulis mempersiapkan presentasi sidang skripsi.
		\item Vania Stephanie dan Khezia Josephine sebagai sahabat penulis yang senantiasa memberikan dukungan, semangat, masukkan yang berguna untuk penulis, serta menghibur penulis selama proses penyusunan skripsi ini terutama saat sedang mengalami kesulitan.
		\item Teman-teman dari Grup MANAYGBILANGWGAGUNA yaitu AK, Otung, Devie, Gilbert, Hoshea, Jeane, Khen, Mark, Matthew, Oyeng, Bebe, Rifo, Rizky, Sean, Vinny, dan WM yang telah menghibur dan mendukung penulis dalam penyusunan skripsi ini.
		\item Teman-teman dari Grup Korea yaitu Ario, Dandy, Fakhry, Felis, Hima, Hizkia, Irvan, Acong, Joshua, Kezia, Edrick, Ocin, Momon, Thoby, Victor, Yona, Yudhis, dan Matthew Ariel sebagai teman seperjuangan di Teknik Informatika UNPAR angkatan 2015 yang telah memberi semangat dan dukungan kepada penulis.
		\item Teman-teman penulis lain yang tidak dapat disebutkan satu persatu. Terima kasih untuk segala dukungannya sehingga skripsi ini dapat diselesaikan dengan baik.
	\end{enumerate}

Akhir kata, semoga skripsi ini dapat bermanfaat bagi pembaca dan dapat menjadi dasar untuk penelitian yang terkait dengan skripsi ini.} 
%=============================================================================

%_____________________________________________________________________________
%=============================================================================
% 								BAGIAN XIV
%=============================================================================
% Tambahkan hyphen (pemenggalan kata) yang anda butuhkan di sini 
%=============================================================================
\hyphenation{ma-te-ma-ti-ka}
\hyphenation{fi-si-ka}
\hyphenation{tek-nik}
\hyphenation{in-for-ma-ti-ka}
%=============================================================================

%_____________________________________________________________________________
%=============================================================================
% 								BAGIAN XV
%=============================================================================
% Tambahkan perintah yang anda buat sendiri di sini 
%=============================================================================
\renewcommand{\vtemplateauthor}{lionov}
\pgfplotsset{compat=newest}

\newcommand{\term}{\textit{term }}
\newcommand{\Term}{\textit{Term }}

\makeatletter
\renewcommand{\ALG@name}{Algoritma}
\makeatother



%=============================================================================
